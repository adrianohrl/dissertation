\chapter[Introdução]{Introdução} \label{cap:introducao}

    \section{Motivação} \label{sec:motivacao}
        Aplicações de robótica onde vários robôs interagem entre si e também com o ambiente em que estão inseridos são chamadas de sistemas multirrobô, do inglês \textit{Multi-robot systems} (MRS). Um sistema multirrobô possui diversas vantagens sobre sistemas com apenas um robô. Entre elas se encontram o ganho de flexibilidade, a simplificação de tarefas complexas e o aumento da eficiência no uso de recursos, de desempenho do sistema como um todo e da robustez através de redundâncias \cite{ref:cao1997cooperative, ref:dudek1996taxonomy, ref:zlot2002multi}. Entretanto, aplicações dessa natureza demandam arquiteturas complexas para o controle da coordenação dos robôs envolvidos e, intrinsecamente, possuem problema de escalabilidade nos processos computacionais e na rede de comunicação. 
        
        Um dos problemas mais desafiadores em aplicações de vários robôs é denominado \textit{alocação de tarefa} (MRTA, acrônimo para \textit{Multi-Robot Task Allocation}), que busca atribuir a execução de um conjunto de tarefas para um grupo de robôs sujeitos à limitações de forma que o desempenho geral do sistema seja otimizado. Esse tipo de problema pode ser resolvido por arquiteturas que se baseiam em modelos de organização que podem ser encontrados no cotidiano. Suas premissas limitam a abrangência de problemas que podem ser resolvidos pela a arquitetura. Com isso, há uma grande quantidade de arquiteturas formuladas. E com o intuito de classificá-las, \citeonline{ref:gerkey2004taxonomy} sugeriu uma taxonomia independente do domínio para a classificação de problemas MRTA a partir da análise de várias delas \cite{ref:parker1998alliance, ref:gerkey2002murdoch, ref:botelho1999m+, ref:werger2000ble, ref:frank2005kuhn, ref:stentz1999fpo, ref:chaimowicz2002dra}. 
        
        Com o advento do ROS (do inglês \textit{Robot Operating System}) \cite{ref:quigley2009ros}, vários sistemas inteligentes puderam ser reutilizados em diversas aplicações de robótica, tais como: localização \cite{ref:li2017kld-samcl}, navegação robótica, gerenciamento de largura de banda \cite{ref:julio2015dynamic}, planejamento e escalonamento de ações e tarefas \cite{ref:fox2003pddl2, ref:manne1960job}, algoritmos de inteligência artificial \cite{ref:adrianohrl2015fuzzy, ref:watkins1992qlearning}, entre outros. Sendo um \textit{middleware} dedicado para aplicações robóticas, ele possibilitou a integração de trabalhos desenvolvidos por equipes distintas de pesquisa em robótica, pois ele simplifica o desenvolvimento de processos e dá suporte à comunicação e interoperabilidade deles. Desta forma, pesquisadores de robótica podem ater-se ao desenvolvimento de projetos dentro da sua especialização, necessitando apenas configurar os demais pacotes para a execução da aplicação. Problemas que anteriormente possuíam difícil solução em termos de \textit{software}, foram simplificados a partir da modularidade proporcionada pelo ROS.
        
        Apesar da vasta existência de arquiteturas de alocação de tarefa para sistemas multirrobôs, houveram poucas tentativas de aproximação genérica delas em projetos baseados em ROS. \textcolor{red}{Isto é, mediante a pesquisa realizada na literatura e no repositório do ROS, não foram encontrados projetos baseados em ROS que se aproximam de uma arquitetura MRTA configurável para qualquer problema MRTA que satisfaça suas premissas. \citeonline{ref:reis2015alliance} mostraram as facilidades que o ROS oferece na implementação da arquitetura ALLIANCE, proposta por \citeonline{ref:parker1998alliance}. Contudo, essa aproximação atende apenas o problema aplicado nesse trabalho. Já em \textit{ auction\_methods\_stack}\footnote{\url{https://github.com/joaoquintas/auction_methods_stack}}, ainda que não possui documentação alguma, está contido um grupo de pacotes que foi desenvolvido em uma versão antiga do ROS e nunca mais foi atualizado.}
        
        Com isso, verifica-se a necessidade de ferramentas que facilitem a utilização de arquiteturas de alocação de tarefa em sistemas multirrobô para o \textit{framework} ROS para incentivar o desenvolvimento de aproximações genéricas dessas arquiteturas.
    
    \section{Objetivos} \label{sec:objetivos}
        Esse trabalho propõe desenvolver um pacote ROS, denominado \textit{rqt\_mrta}, que facilite a utilização de arquiteturas de alocação de tarefa no ROS para sistemas multirrobô. Esse pacote fornece uma interface gráfica que foi desenvolvida com o intuito de disponibilizar serviços para dois tipos de clientes: (1) desenvolvedor e (2) usuário de arquitetura MRTA. Seus serviços são:
        
        \begin{itemize}
            \item Cadastro de novas arquiteturas no ROS para seu uso na solução de problemas de alocação de tarefa em sistemas multirrobô;
            \item Criação de novos projetos contendo a definição de um problema de alocação de tarefa;
            \item Configuração da arquitetura escolhida para resolver o problema MRTA;
            \item Armazenamento dos dados de configuração no projeto criado;
            \item Monitoramento da comunicação dos robôs do sistema no ROS em tempo de execução;
            \item Monitoramento das atividades dos robôs no sistema em tempo de execução.
        \end{itemize}
        
        Além disso, será apresentado neste trabalho uma aproximação genérica da arquitetura Alliance, a qual foi desenvolvida em um pacote ROS chamado \textit{alliance}.
        
    \section{Contribuições} \label{sec:contribuicoes}
        A partir da elaboração deste trabalho, os seguintes pacotes baseados em ROS foram obtidos e disponibilizados para a comunidade ROS:
        
        \begin{itemize}
            \item \textbf{\textit{rqt\_mrta}}\footnote{\url{http://wiki.ros.org/rqt_mrta}}: uma interface gráfica usuário que facilita a utilização de arquiteturas que resolvem o problema de alocação de tarefa em um sistema multirrobô.
            \item \textbf{\textit{alliance}}\footnote{\url{http://wiki.ros.org/alliance}}: uma aproximação genérica da arquitetura Alliance.
            \item \textbf{\textit{alliance\_msgs}}\footnote{\url{http://wiki.ros.org/alliance_msgs}}: contém definição das mensagens utilizadas na comunicação entre os robôs na arquitetura Alliance pelo pacote \textit{alliance}.
            \item \textbf{\textit{rqt\_alliance}}\footnote{\url{http://wiki.ros.org/rqt_alliance}}: uma interface gráfica de usuário que monitora as variáveis de motivação dos robôs em um sistema que utiliza o pacote \textit{alliance} para a alocação de tarefas.
        \end{itemize}
    
    \section{Estrutura do Trabalho} \label{sec:estrutura}
        No capítulo \ref{cap:revisao}, ... 
        Prosseguindo para o capítulo \ref{cap:desenvolvimento}, ...
        A seguir, ..., no capítulo \ref{cap:resultados}.
        Por fim, o capítulo \ref{cap:conclusao} ...