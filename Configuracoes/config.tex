% Pacotes fundamentais 
\usepackage[brazil]{babel}
\usepackage{cmap}			    % Mapear caracteres especiais no PDF
\usepackage{lmodern}			% Usa a fonte Latin Modern			
\usepackage[T1]{fontenc}		% Selecao de codigos de fonte.
\usepackage[utf8]{inputenc}		% Codificacao do documento (conversão automática dos acentos)
\usepackage{lastpage}			% Usado pela Ficha catalográfica
\usepackage{indentfirst}		% Indenta o primeiro parágrafo de cada seção.
\usepackage{color}			    % Controle das cores
\usepackage{graphicx}			% Inclusão de gráficos
%\usepackage{epstopdf}          % Apenas no windows
\usepackage{bmpsize}
\usepackage[cmex10]{amsmath}    % Formulas matemáticas
\usepackage{amsfonts,amssymb,latexsym}
%\usepackage{amscls}
\usepackage{subfig}
\usepackage[siunitx]{circuitikz}
\usepackage{longtable}
\usepackage{tabularx}
\usepackage{array,booktabs}
\usepackage{multirow}
\usepackage{pgfplots}
\pgfplotsset{compat=1.14}
%fazer uma pagina em retrato
%\usepackage{geometry}
\usepackage{pdflscape}
\usepackage{nomencl}
\makenomenclature
\usepackage{textcase}

% Pacotes adicionais, usados apenas no âmbito do Modelo Canônico do abnteX2
\usepackage{lipsum}				% para geração de dummy text

% Pacotes de citações
\usepackage[brazilian,hyperpageref]{backref}	% Paginas com as citações na bibl
\usepackage[alf, bibjustif]{abntex2cite}		% Citações padrão ABNT
\hyphenation{di-mi-nu-i-ção}
\hyphenation{ha-lo-im-plan-ta-dos}

% Pacotes para plotar gráficos 
\usepackage{pgfplots}
\pgfplotsset{compat=newest}
\usepackage{pgfplotstable}
\usepackage{siunitx}
\usetikzlibrary{pgfplots.groupplots}

\usepackage{svg}

\usepackage{hyperref}

\usepackage{amssymb}	 % qed
\usepackage{amsthm}      % Teoremas
\usepackage{thmtools}    % Front end para amsthm (\declaretheorem)

\declaretheorem[style=definition,name=Definição,parent=chapter,qed=\textemdash]{definicao}
\declaretheorem[style=plain,name=Teorema,qed=\textnormal{\textemdash}]{teorema}
\declaretheorem[style=plain,name=Axioma,qed=\textnormal{\textemdash}]{axioma}

\usepackage{listings,lstautogobble}

% A numeração de figuras e tabelas deve ser contínua em todo o documento (ABNT NBR 14724:2011 seções 5.8 e 5.9). Porém, caso deseje alterar esse comportamento para numeração por capítulos, por exemplo, use:
%\counterwithin{figure}{section}
%\counterwithin{table}{section}

% CONFIGURAÇÕES DE PACOTES
% Configurações do pacote backref
% Usado sem a opção hyperpageref de backref
%\renewcommand{\backrefpagesname}{Citado na(s) página(s):~}
% Texto padrão antes do número das páginas
%\renewcommand{\backref}{}
% Define os textos da citação
%\renewcommand*{\backrefalt}[4]{
%	\ifcase #1 %
%		Nenhuma citação no texto.%
%	\or
%		Citado na página #2.%
%	\else
%		Citado #1 vezes nas páginas #2.%
%	\fi}%

% Configurações de aparência do PDF final
% alterando o aspecto da cor azul
\definecolor{blue}{RGB}{41,5,195}

% informações do PDF
\makeatletter
\hypersetup{
     	%pagebackref=true,
		pdftitle={\@title}, 
		pdfauthor={\@author},
    	pdfsubject={\imprimirpreambulo},
	    pdfcreator={LaTeX with abnTeX2},
		pdfkeywords={abnt}{latex}{abntex}{abntex2}{trabalho acadêmico}, 
		colorlinks=true,       		% false: boxed links; true: colored links
    	linkcolor=blue,          	% color of internal links
    	citecolor=blue,        		% color of links to bibliography
    	filecolor=magenta,      	% color of file links
		urlcolor=blue,
		bookmarksdepth=4
}
\makeatother

% Espaçamentos entre linhas e parágrafos 
% Retira espaço extra obsoleto entre as frases.
\frenchspacing 

% O tamanho do parágrafo é dado por:
\setlength{\parindent}{1.3cm}

% Controle do espaçamento entre um parágrafo e outro:
\setlength{\parskip}{0.2cm}  % tente também \onelineskip

% compila o indice
\makeindex