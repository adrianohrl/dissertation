\chapter[Conclusão e Trabalhos Futuros]{Conclusão e Trabalhos Futuros} \label{cap:conclusao}

    \section{Conclusão}
        Este trabalho apresentou o desenvolvimento de uma interface gráfica de usuário integrada com o \textit{framework} ROS com o intuito de facilitar a parametrização de arquiteturas de alocação de tarefa em aplicações com múltiplos robôs. A interface desenvolvida fornece serviços de criação e supervisão de aplicações multirrobô no ROS. Durante o processo de criação de aplicação é solicitado o preenchimento correto dos parâmetros da arquitetura selecionada. 
        
        Primeiramente, este trabalho apresentou um estudo sobre sistemas multirrobô, identificando suas vantagens perante um sistema de um único robô, bem como, suas características. Em seguida, foi revisado o problema de alocação de tarefa nesses sistemas e como eles podem ser classificados. Verificou-se a existência de várias arquiteturas que visam resolver estes problemas. Uma análise do ROS foi feita apresentando os seus conceitos básicos e o desenvolvimento de interfaces gráficas de usuário integrada com o \textit{framework}. Por fim, pesquisas relacionadas ao desenvolvimento de arquiteturas MRTA mostraram que existem poucas aproximações genéricas implementadas no ROS e que normalmente elas não possuem boa documentação.
        
        O capítulo de desenvolvimento mostrou como são estruturados os arquivos de configuração de arquitetura e de  aplicação. Em cada caso, foi apresentado o processo de cadastro de arquiteturas e aplicações de modo que o pacote \textit{rqt\_mrta} identificá-las. Posteriormente, foram explicadas as camadas de modelo, visualização e controle do projeto \textit{rqt\_mrta}. 
    
        Uma implementação genérica da arquitetura ALLIANCE foi desenvolvida no pacote \textit{alliance} para o teste do pacote \textit{rqt\_mrta}. Para que esta arquitetura pudesse ser utilizada por aplicações multirrobô, foi criado o arquivo de configuração da arquitetura do \textit{alliance}, exportando a localização deste arquivo através do seu manifesto. Em seguida, verificou-se que o cadastro desta arquitetura foi realizado com sucesso.
        
        Uma aplicação com vários robôs em um sistema de patrulhamento foi criada e simulada em conjunto com o pacote \textit{alliance} para validar as funcionalidades do \textit{plugin rqt\_mrta}. Durante a sua criação (via \textit{wizard}) foram inseridos os dados gerais da aplicação, selecionada a arquitetura do pacote \textit{alliance}, indicados os robôs do sistema e, por fim, preenchidos os parâmetros obrigatórios da arquitetura selecionada. Ao final deste procedimento, foi iniciado um novo \textit{workspace} do ROS onde foi criado um novo pacote ROS contendo os arquivos gerados a partir do preenchimento dos dados da aplicação. Na sequência, os arquivos gerados foram examinados e utilizados na inicialização dos nós necessários para a execução da arquitetura selecionada. Foi averiguado que a comunicação entre os nós da arquitetura estava correta.
        
        Com o auxilio de um simulador, a aplicação foi executada juntamente com os nós da arquitetura do pacote \textit{alliance}. Durante sua execução, os recursos de supervisão gráfica do \textit{plugin rqt\_mrta} foi analisado.
        
        Concluindo, o processo de criação de aplicação multirrobô através do \textit{rqt\_mrta} mostrou que, após a configuração e cadastro da arquitetura, sua parametrização para o uso em aplicações torna-se mais intuitiva devido a existência de uma documentação mínima. %Além disso, o \textit{plugin} facilita a criação dos arquivos exigidos pela arquitetura.
        
    \section{Trabalhos Futuros}
        Uma melhoria significativa desta aplicação seria possibilitar a inscrição de novas arquiteturas devidamente configuradas no repositório deste projeto. De modo que, sempre que uma nova inscrição ocorresse no \textit{rqt\_mrta}, seria solicitada uma nova atualização a todos usuários do pacote \textit{rqt\_mrta} para baixar e instalar a nova arquitetura inscrita. Isso eliminaria a necessidade de procurar novas abordagens de arquitetura por seus usuários e ajudaria o desenvolvedor de arquiteturas a impulsionar seu projeto na comunidade ROS.
        
        Enfim, a adição de mais alguns recurso no \textit{plugin} resultaria em um maior envolvimento do desenvolvedores e usuários de arquiteturas, como:
        
        \begin{itemize}
            \item a comparação do uso de arquiteturas diferentes em uma mesma aplicação para a análise de desempenho;
            
            \item geração de relatórios gráficos e numéricos durante a execução da aplicação;
            
            \item inicialização e parada da aplicação através do \textit{plugin};
            
            \item serviços de agendamento;
            
            \item simulação de tarefas para validação da arquitetura em desenvolvimento;
            
            \item entre outros.
        \end{itemize}
        
        