\chapter[ROS]{ROS} \label{cap:ros}
    
    Acrônimo para \textit{Robot Operting System} \cite{ref:quigley2009ros}, o ROS é uma \textit{framework} que tem incentivado a comunidade de pesquisadores em robótica a trabalhar conjuntamente desde seu lançamento.
    
    Primeiramente, ROS é flexível. Um projeto atômico baseado em ROS, denominado pacote, pode ser desenvolvido em diversas linguagens de programação. Deste modo, seus desenvolvedores podem tirar proveito das vantagens que cada linguagem suportada tem, sejam elas eficiência em tempo de execução, confiabilidade, recursos, síntaxe, semântica ou documentação existente. Atualmente, as linguagens de programação suportadas são C++, Python e Lisp. As linguagens Java e Lua ainda estão em fase de desenvolvimento.
    
    Projetos de robótica possuem rotinas que poderia ser reutilizadas em outros projetos. Por esta razão, ROS é também modular, pois pacotes configuráveis existentes podem ser combinados para realizar uma aplicação especifica. Várias bibliotecas externas já foram adaptadas para ser usadas no ROS: aruco\footnote{\url{http://wiki.ros.org/ar_sys}}, gmapping\footnote{\url{http://wiki.ros.org/gmapping}}, interfaces de programação para aplicações de robôs\footnote{\url{http://wiki.ros.org/Robots}}, sensores\footnote{\url{http://wiki.ros.org/Sensors}} e simuladores\footnote{\url{http://wiki.ros.org/gazebo}}, planejadores\footnote{\url{http://kcl-planning.github.io/ROSPlan/}}, reconhecimento de voz\footnote{\url{http://wiki.ros.org/Sensors\#Audio_.2BAC8_Speech_Recognition}}, entre outros. Isso evidencia que os usuários de ROS podem focar no desenvolvimento de pesquisa de sua área e contribuir da melhor forma com essa comunidade.
    
    Em adição, ROS disponibiliza diversas ferramentas para auxiliar no desenvolvimento de projetos e, também, verificar o funcionamento de aplicação. Suas ferramentas típicas são: \textit{get} e \textit{set} de parâmetros de configuração, vizualização da topologia de conexão \textit{peer-to-peer}, medição de utilização de banda, gráficos dos dados de mensagem e outras mais. É altamente recomendado o uso dessas ferramentas para garantir a estabilidade e confiança dos pacotes desenvolvidos, que normalmente têm alta complexidade.
    
    Uma lacuna que antes existia na nova geração de aplicações robóticas foi preenchida com o lançamento do ROS. Como um fornecedor de serviços de \textit{middleware}, ele (1) simplifica o desenvolvimento de processos, (2) suporta comunicação e interoperabilidade, (3) oferece e facilita serviços frequentemente utilizados em robótica e, ainda, oferece (4) utilização eficiente dos seus recursos disponíveis, (5) abstrações heterogênicas e (6) descoberta e configuração automática de recursos \cite{ref:quigley2009ros}. No intuito de cobrir todas exigências de um \textit{middleware}, ROS 2.0 tenta dar suporte à componentes embarcados e dispositivos de baixo recurso.
    
    Como resultado, 
    
    Over the years, it has been noticed that:
	(1) the number of contributors (academic researchers and industry) and projects have increased;
	(2) the applications have became more sophisticated;
	(3) the degree of difficulty of solved problems has ??? in different areas of robotics field;
	(4) the robotic industry has been more interested to contribute.
	
	\textcolor{red}{citar exemplo de middlewares para robótica}
    
    \textcolor{red}{falar sobre o conteúdo deste capítulo}
    
    \section{Conceitos Básicos} \label{sec:ros_conceitos}
    
    