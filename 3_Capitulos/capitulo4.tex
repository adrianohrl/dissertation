\chapter[Recursos]{Recursos} \label{cap:recursos}

    Existem duas classes de recursos, basicamente: aqueles que podem ser usados novamente, os quais são denominados \textit{recursos reusáveis}; e aqueles que são descartáveis, denominados \textit{recursos consumíveis} \cite{ref:ghallab2004automated}. Cada classe de recurso é caracterizada por um \textit{perfil}. O perfil de um recurso $r$ é uma função do tempo, $z_r(t)$, que define sua quantidade instantânea.
    
    A seguir, cada classe de recurso será definida, bem como, será detalhado o perfil de cada uma delas.
    
    % 1a seção do capítulo 4
    \section{Recursos Reutilizáveis} \label{sec:recurso_reusavel}
    
        Um recurso reusável é ``emprestado'' por uma tarefa durante sua execução. E quando ela é interrompida ou finalizada, esse recurso é liberado sem alterações. Assim, um recurso reusável $r$ tem uma capacidade total $Q_r$ e uma quantidade corrente $z_r(t) \in [0; Q_r]$. 
        
        Seja uma tarefa $a$ que requer durante a sua execução uma quantidade $q$ do recurso $r$. Ao ser iniciada, no tempo $s(a)$, a quantidade corrente de $r$ é diminuída em um montante $q$. Entretanto, quando $a$ é finalizada, no tempo $e(a)$, a quantidade corrente de $r$ é aumentada pelo mesmo montante, $q$. Portanto, um recurso reusável possui um perfil característico conforme mostra a Figura \ref{fig:recurso_reusavel}. \emph{\color{red} (explicar Figura \ref{fig:recurso_reusavel} melhor (usar a palavra ``evolução'').)}
        
        \begin{figure}
            \centering
            % Graphic for TeX using PGF
% Title: /home/adrianohrl/Documents/mastering/dissertation/figures/reusable_resource.dia
% Creator: Dia v0.97.2
% CreationDate: Fri Oct 13 18:13:22 2017
% For: adrianohrl
% \usepackage{tikz}
% The following commands are not supported in PSTricks at present
% We define them conditionally, so when they are implemented,
% this pgf file will use them.
\ifx\du\undefined
  \newlength{\du}
\fi
\setlength{\du}{15\unitlength}
\begin{tikzpicture}
\pgftransformxscale{1.000000}
\pgftransformyscale{-1.000000}
\definecolor{dialinecolor}{rgb}{0.000000, 0.000000, 0.000000}
\pgfsetstrokecolor{dialinecolor}
\definecolor{dialinecolor}{rgb}{1.000000, 1.000000, 1.000000}
\pgfsetfillcolor{dialinecolor}
\pgfsetlinewidth{0.050000\du}
\pgfsetdash{{1.000000\du}{1.000000\du}}{0\du}
\pgfsetdash{{0.300000\du}{0.300000\du}}{0\du}
\pgfsetbuttcap
{
\definecolor{dialinecolor}{rgb}{0.000000, 0.000000, 0.000000}
\pgfsetfillcolor{dialinecolor}
% was here!!!
\definecolor{dialinecolor}{rgb}{0.000000, 0.000000, 0.000000}
\pgfsetstrokecolor{dialinecolor}
\draw (27.500000\du,17.500000\du)--(29.000000\du,17.500000\du);
}
\pgfsetlinewidth{0.050000\du}
\pgfsetdash{{0.300000\du}{0.300000\du}}{0\du}
\pgfsetdash{{0.300000\du}{0.300000\du}}{0\du}
\pgfsetbuttcap
{
\definecolor{dialinecolor}{rgb}{0.000000, 0.000000, 0.000000}
\pgfsetfillcolor{dialinecolor}
% was here!!!
\pgfsetarrowsstart{stealth}
\pgfsetarrowsend{stealth}
\definecolor{dialinecolor}{rgb}{0.000000, 0.000000, 0.000000}
\pgfsetstrokecolor{dialinecolor}
\draw (28.000000\du,16.000000\du)--(28.000000\du,17.500000\du);
}
% setfont left to latex
\definecolor{dialinecolor}{rgb}{0.000000, 0.000000, 0.000000}
\pgfsetstrokecolor{dialinecolor}
\node at (27.000000\du,16.971250\du){$q_2$};
\pgfsetlinewidth{0.050000\du}
\pgfsetdash{{0.300000\du}{0.300000\du}}{0\du}
\pgfsetdash{{0.300000\du}{0.300000\du}}{0\du}
\pgfsetbuttcap
{
\definecolor{dialinecolor}{rgb}{0.000000, 0.000000, 0.000000}
\pgfsetfillcolor{dialinecolor}
% was here!!!
\definecolor{dialinecolor}{rgb}{0.000000, 0.000000, 0.000000}
\pgfsetstrokecolor{dialinecolor}
\draw (26.000000\du,16.000000\du)--(27.500000\du,16.000000\du);
}
\pgfsetlinewidth{0.050000\du}
\pgfsetdash{{0.300000\du}{0.300000\du}}{0\du}
\pgfsetdash{{0.300000\du}{0.300000\du}}{0\du}
\pgfsetbuttcap
{
\definecolor{dialinecolor}{rgb}{0.000000, 0.000000, 0.000000}
\pgfsetfillcolor{dialinecolor}
% was here!!!
\pgfsetarrowsstart{stealth}
\pgfsetarrowsend{stealth}
\definecolor{dialinecolor}{rgb}{0.000000, 0.000000, 0.000000}
\pgfsetstrokecolor{dialinecolor}
\draw (26.500000\du,12.500000\du)--(26.500000\du,16.000000\du);
}
% setfont left to latex
\definecolor{dialinecolor}{rgb}{0.000000, 0.000000, 0.000000}
\pgfsetstrokecolor{dialinecolor}
\node at (25.500000\du,14.471250\du){$q_1$};
\pgfsetlinewidth{0.050000\du}
\pgfsetdash{{0.300000\du}{0.300000\du}}{0\du}
\pgfsetdash{{0.300000\du}{0.300000\du}}{0\du}
\pgfsetbuttcap
{
\definecolor{dialinecolor}{rgb}{0.000000, 0.000000, 0.000000}
\pgfsetfillcolor{dialinecolor}
% was here!!!
\definecolor{dialinecolor}{rgb}{0.000000, 0.000000, 0.000000}
\pgfsetstrokecolor{dialinecolor}
\draw (24.000000\du,12.500000\du)--(36.500000\du,12.500000\du);
}
% setfont left to latex
\definecolor{dialinecolor}{rgb}{0.000000, 0.000000, 0.000000}
\pgfsetstrokecolor{dialinecolor}
\node at (23.000000\du,12.721250\du){$Q_r$};
\pgfsetlinewidth{0.100000\du}
\pgfsetdash{}{0pt}
\pgfsetdash{}{0pt}
\pgfsetbuttcap
{
\definecolor{dialinecolor}{rgb}{0.000000, 0.000000, 0.000000}
\pgfsetfillcolor{dialinecolor}
% was here!!!
\pgfsetarrowsend{stealth}
\definecolor{dialinecolor}{rgb}{0.000000, 0.000000, 0.000000}
\pgfsetstrokecolor{dialinecolor}
\draw (24.000000\du,21.000000\du)--(24.000000\du,11.000000\du);
}
\pgfsetlinewidth{0.100000\du}
\pgfsetdash{}{0pt}
\pgfsetdash{}{0pt}
\pgfsetbuttcap
{
\definecolor{dialinecolor}{rgb}{0.000000, 0.000000, 0.000000}
\pgfsetfillcolor{dialinecolor}
% was here!!!
\pgfsetarrowsend{stealth}
\definecolor{dialinecolor}{rgb}{0.000000, 0.000000, 0.000000}
\pgfsetstrokecolor{dialinecolor}
\draw (23.000000\du,20.000000\du)--(38.000000\du,20.000000\du);
}
% setfont left to latex
\definecolor{dialinecolor}{rgb}{0.000000, 0.000000, 0.000000}
\pgfsetstrokecolor{dialinecolor}
\node at (24.000000\du,10.221250\du){$z_r$};
% setfont left to latex
\definecolor{dialinecolor}{rgb}{0.000000, 0.000000, 0.000000}
\pgfsetstrokecolor{dialinecolor}
\node at (39.000000\du,20.221250\du){$t$};
\pgfsetlinewidth{0.150000\du}
\pgfsetdash{}{0pt}
\pgfsetdash{}{0pt}
\pgfsetbuttcap
{
\definecolor{dialinecolor}{rgb}{0.000000, 0.000000, 0.000000}
\pgfsetfillcolor{dialinecolor}
% was here!!!
\definecolor{dialinecolor}{rgb}{0.000000, 0.000000, 0.000000}
\pgfsetstrokecolor{dialinecolor}
\draw (23.960000\du,12.500000\du)--(27.570000\du,12.500000\du);
}
\pgfsetlinewidth{0.150000\du}
\pgfsetdash{}{0pt}
\pgfsetdash{}{0pt}
\pgfsetbuttcap
{
\definecolor{dialinecolor}{rgb}{0.000000, 0.000000, 0.000000}
\pgfsetfillcolor{dialinecolor}
% was here!!!
\definecolor{dialinecolor}{rgb}{0.000000, 0.000000, 0.000000}
\pgfsetstrokecolor{dialinecolor}
\draw (27.500000\du,12.430000\du)--(27.500000\du,16.070000\du);
}
\pgfsetlinewidth{0.150000\du}
\pgfsetdash{}{0pt}
\pgfsetdash{}{0pt}
\pgfsetbuttcap
{
\definecolor{dialinecolor}{rgb}{0.000000, 0.000000, 0.000000}
\pgfsetfillcolor{dialinecolor}
% was here!!!
\definecolor{dialinecolor}{rgb}{0.000000, 0.000000, 0.000000}
\pgfsetstrokecolor{dialinecolor}
\draw (27.430000\du,16.000000\du)--(29.070000\du,16.000000\du);
}
\pgfsetlinewidth{0.150000\du}
\pgfsetdash{}{0pt}
\pgfsetdash{}{0pt}
\pgfsetbuttcap
{
\definecolor{dialinecolor}{rgb}{0.000000, 0.000000, 0.000000}
\pgfsetfillcolor{dialinecolor}
% was here!!!
\definecolor{dialinecolor}{rgb}{0.000000, 0.000000, 0.000000}
\pgfsetstrokecolor{dialinecolor}
\draw (29.000000\du,15.930000\du)--(29.000000\du,17.570000\du);
}
\pgfsetlinewidth{0.150000\du}
\pgfsetdash{}{0pt}
\pgfsetdash{}{0pt}
\pgfsetbuttcap
{
\definecolor{dialinecolor}{rgb}{0.000000, 0.000000, 0.000000}
\pgfsetfillcolor{dialinecolor}
% was here!!!
\definecolor{dialinecolor}{rgb}{0.000000, 0.000000, 0.000000}
\pgfsetstrokecolor{dialinecolor}
\draw (28.930000\du,17.500000\du)--(32.570000\du,17.500000\du);
}
\pgfsetlinewidth{0.150000\du}
\pgfsetdash{}{0pt}
\pgfsetdash{}{0pt}
\pgfsetbuttcap
{
\definecolor{dialinecolor}{rgb}{0.000000, 0.000000, 0.000000}
\pgfsetfillcolor{dialinecolor}
% was here!!!
\definecolor{dialinecolor}{rgb}{0.000000, 0.000000, 0.000000}
\pgfsetstrokecolor{dialinecolor}
\draw (32.500000\du,13.930000\du)--(32.500000\du,17.570000\du);
}
\pgfsetlinewidth{0.150000\du}
\pgfsetdash{}{0pt}
\pgfsetdash{}{0pt}
\pgfsetbuttcap
{
\definecolor{dialinecolor}{rgb}{0.000000, 0.000000, 0.000000}
\pgfsetfillcolor{dialinecolor}
% was here!!!
\definecolor{dialinecolor}{rgb}{0.000000, 0.000000, 0.000000}
\pgfsetstrokecolor{dialinecolor}
\draw (32.430000\du,14.000000\du)--(33.570000\du,14.000000\du);
}
\pgfsetlinewidth{0.150000\du}
\pgfsetdash{}{0pt}
\pgfsetdash{}{0pt}
\pgfsetbuttcap
{
\definecolor{dialinecolor}{rgb}{0.000000, 0.000000, 0.000000}
\pgfsetfillcolor{dialinecolor}
% was here!!!
\definecolor{dialinecolor}{rgb}{0.000000, 0.000000, 0.000000}
\pgfsetstrokecolor{dialinecolor}
\draw (33.500000\du,12.430000\du)--(33.500000\du,14.070000\du);
}
\pgfsetlinewidth{0.150000\du}
\pgfsetdash{}{0pt}
\pgfsetdash{}{0pt}
\pgfsetbuttcap
{
\definecolor{dialinecolor}{rgb}{0.000000, 0.000000, 0.000000}
\pgfsetfillcolor{dialinecolor}
% was here!!!
\definecolor{dialinecolor}{rgb}{0.000000, 0.000000, 0.000000}
\pgfsetstrokecolor{dialinecolor}
\draw (33.430000\du,12.500000\du)--(36.570000\du,12.500000\du);
}
\pgfsetlinewidth{0.100000\du}
\pgfsetdash{}{0pt}
\pgfsetdash{}{0pt}
\pgfsetmiterjoin
\definecolor{dialinecolor}{rgb}{0.749020, 0.749020, 0.749020}
\pgfsetfillcolor{dialinecolor}
\fill (27.500000\du,21.500000\du)--(27.500000\du,22.000000\du)--(32.500000\du,22.000000\du)--(32.500000\du,21.500000\du)--cycle;
\definecolor{dialinecolor}{rgb}{0.000000, 0.000000, 0.000000}
\pgfsetstrokecolor{dialinecolor}
\draw (27.500000\du,21.500000\du)--(27.500000\du,22.000000\du)--(32.500000\du,22.000000\du)--(32.500000\du,21.500000\du)--cycle;
% setfont left to latex
\definecolor{dialinecolor}{rgb}{0.000000, 0.000000, 0.000000}
\pgfsetstrokecolor{dialinecolor}
\node at (30.000000\du,22.711906\du){$requerer(t_1, r, q_1)$};
\pgfsetlinewidth{0.100000\du}
\pgfsetdash{}{0pt}
\pgfsetdash{}{0pt}
\pgfsetmiterjoin
\definecolor{dialinecolor}{rgb}{0.749020, 0.749020, 0.749020}
\pgfsetfillcolor{dialinecolor}
\fill (29.000000\du,23.500000\du)--(29.000000\du,24.000000\du)--(33.500000\du,24.000000\du)--(33.500000\du,23.500000\du)--cycle;
\definecolor{dialinecolor}{rgb}{0.000000, 0.000000, 0.000000}
\pgfsetstrokecolor{dialinecolor}
\draw (29.000000\du,23.500000\du)--(29.000000\du,24.000000\du)--(33.500000\du,24.000000\du)--(33.500000\du,23.500000\du)--cycle;
% setfont left to latex
\definecolor{dialinecolor}{rgb}{0.000000, 0.000000, 0.000000}
\pgfsetstrokecolor{dialinecolor}
\node at (31.250000\du,24.711906\du){$requerer(t_2, r, q_2)$};
\end{tikzpicture}

            \caption{Perfil característico de um recurso reusável.} \label{fig:recurso_reusavel}
        \end{figure}
    
    % 2a seção do capítulo 4
    \section{Recursos Consumíveis} \label{sec:recurso_consumivel}
    
        Um recurso consumível pode ser produzido ou consumido durante a execução de uma tarefa. Esta classe de recurso pode ser modelada como um reservatório de capacidade máxima limitada $Q_r$ e nível (quantidade) corrente $z_r(t) \in [0; Q_r]$.
        
        Seja, pois, uma tarefa $a$ que produz durante a sua execução uma quantidade $q$ do recurso $r$. Quando iniciada, em $s(a)$, aumenta um montante $q$ do seu nível $z_r(t)$ ao longo do tempo. Essa produção é modelada por uma função dependente do tempo, crescente no intervalo temporal $[s(a); e(a)]$. Seja, agora, uma tarefa $a$ que consome durante a sua execução uma quantidade $q$ do recurso $r$. Quando iniciada, no instante $s(a)$, reduz um montante $q$ do seu nível $z_r(t)$ ao longo do tempo. Essa redução/consumo é modelada como uma função do tempo, decrescente no intervalo de tempo $[s(a); e(a)]$. Portanto, um recurso consumível possui um perfil característico conforme mostrado na Figura \ref{fig:recurso_consumivel}. \emph{\color{red} (explicar Figura \ref{fig:recurso_consumivel} melhor (usar a palavra ``evolução'').)}
        
        \begin{figure}
            \centering
            % Graphic for TeX using PGF
% Title: /home/adrianohrl/Documents/mastering/dissertation/figures/consumable_resource.dia
% Creator: Dia v0.97.2
% CreationDate: Fri Oct 13 18:11:57 2017
% For: adrianohrl
% \usepackage{tikz}
% The following commands are not supported in PSTricks at present
% We define them conditionally, so when they are implemented,
% this pgf file will use them.
\ifx\du\undefined
  \newlength{\du}
\fi
\setlength{\du}{15\unitlength}
\begin{tikzpicture}
\pgftransformxscale{1.000000}
\pgftransformyscale{-1.000000}
\definecolor{dialinecolor}{rgb}{0.000000, 0.000000, 0.000000}
\pgfsetstrokecolor{dialinecolor}
\definecolor{dialinecolor}{rgb}{1.000000, 1.000000, 1.000000}
\pgfsetfillcolor{dialinecolor}
\pgfsetlinewidth{0.100000\du}
\pgfsetdash{}{0pt}
\pgfsetdash{}{0pt}
\pgfsetbuttcap
{
\definecolor{dialinecolor}{rgb}{0.000000, 0.000000, 0.000000}
\pgfsetfillcolor{dialinecolor}
% was here!!!
\pgfsetarrowsend{stealth}
\definecolor{dialinecolor}{rgb}{0.000000, 0.000000, 0.000000}
\pgfsetstrokecolor{dialinecolor}
\draw (15.000000\du,21.000000\du)--(15.000000\du,11.000000\du);
}
\pgfsetlinewidth{0.100000\du}
\pgfsetdash{}{0pt}
\pgfsetdash{}{0pt}
\pgfsetbuttcap
{
\definecolor{dialinecolor}{rgb}{0.000000, 0.000000, 0.000000}
\pgfsetfillcolor{dialinecolor}
% was here!!!
\pgfsetarrowsend{stealth}
\definecolor{dialinecolor}{rgb}{0.000000, 0.000000, 0.000000}
\pgfsetstrokecolor{dialinecolor}
\draw (14.000000\du,20.000000\du)--(29.000000\du,20.000000\du);
}
% setfont left to latex
\definecolor{dialinecolor}{rgb}{0.000000, 0.000000, 0.000000}
\pgfsetstrokecolor{dialinecolor}
\node at (15.000000\du,10.221250\du){$z_r$};
% setfont left to latex
\definecolor{dialinecolor}{rgb}{0.000000, 0.000000, 0.000000}
\pgfsetstrokecolor{dialinecolor}
\node at (30.000000\du,20.221250\du){$t$};
\pgfsetlinewidth{0.050000\du}
\pgfsetdash{{1.000000\du}{1.000000\du}}{0\du}
\pgfsetdash{{0.300000\du}{0.300000\du}}{0\du}
\pgfsetbuttcap
{
\definecolor{dialinecolor}{rgb}{0.000000, 0.000000, 0.000000}
\pgfsetfillcolor{dialinecolor}
% was here!!!
\definecolor{dialinecolor}{rgb}{0.000000, 0.000000, 0.000000}
\pgfsetstrokecolor{dialinecolor}
\draw (16.500000\du,16.000000\du)--(18.000000\du,16.000000\du);
}
\pgfsetlinewidth{0.050000\du}
\pgfsetdash{{0.300000\du}{0.300000\du}}{0\du}
\pgfsetdash{{0.300000\du}{0.300000\du}}{0\du}
\pgfsetbuttcap
{
\definecolor{dialinecolor}{rgb}{0.000000, 0.000000, 0.000000}
\pgfsetfillcolor{dialinecolor}
% was here!!!
\pgfsetarrowsstart{stealth}
\pgfsetarrowsend{stealth}
\definecolor{dialinecolor}{rgb}{0.000000, 0.000000, 0.000000}
\pgfsetstrokecolor{dialinecolor}
\draw (17.000000\du,13.500000\du)--(17.000000\du,16.000000\du);
}
% setfont left to latex
\definecolor{dialinecolor}{rgb}{0.000000, 0.000000, 0.000000}
\pgfsetstrokecolor{dialinecolor}
\node at (16.000000\du,14.971250\du){$q_1$};
\pgfsetlinewidth{0.050000\du}
\pgfsetdash{{0.300000\du}{0.300000\du}}{0\du}
\pgfsetdash{{0.300000\du}{0.300000\du}}{0\du}
\pgfsetbuttcap
{
\definecolor{dialinecolor}{rgb}{0.000000, 0.000000, 0.000000}
\pgfsetfillcolor{dialinecolor}
% was here!!!
\definecolor{dialinecolor}{rgb}{0.000000, 0.000000, 0.000000}
\pgfsetstrokecolor{dialinecolor}
\draw (18.000000\du,17.500000\du)--(19.500000\du,17.500000\du);
}
\pgfsetlinewidth{0.050000\du}
\pgfsetdash{{0.300000\du}{0.300000\du}}{0\du}
\pgfsetdash{{0.300000\du}{0.300000\du}}{0\du}
\pgfsetbuttcap
{
\definecolor{dialinecolor}{rgb}{0.000000, 0.000000, 0.000000}
\pgfsetfillcolor{dialinecolor}
% was here!!!
\pgfsetarrowsstart{stealth}
\pgfsetarrowsend{stealth}
\definecolor{dialinecolor}{rgb}{0.000000, 0.000000, 0.000000}
\pgfsetstrokecolor{dialinecolor}
\draw (18.500000\du,16.000000\du)--(18.500000\du,17.500000\du);
}
% setfont left to latex
\definecolor{dialinecolor}{rgb}{0.000000, 0.000000, 0.000000}
\pgfsetstrokecolor{dialinecolor}
\node at (17.500000\du,16.971250\du){$q_2$};
\pgfsetlinewidth{0.050000\du}
\pgfsetdash{{0.300000\du}{0.300000\du}}{0\du}
\pgfsetdash{{0.300000\du}{0.300000\du}}{0\du}
\pgfsetbuttcap
{
\definecolor{dialinecolor}{rgb}{0.000000, 0.000000, 0.000000}
\pgfsetfillcolor{dialinecolor}
% was here!!!
\definecolor{dialinecolor}{rgb}{0.000000, 0.000000, 0.000000}
\pgfsetstrokecolor{dialinecolor}
\draw (26.000000\du,17.500000\du)--(27.500000\du,17.500000\du);
}
\pgfsetlinewidth{0.050000\du}
\pgfsetdash{{0.300000\du}{0.300000\du}}{0\du}
\pgfsetdash{{0.300000\du}{0.300000\du}}{0\du}
\pgfsetbuttcap
{
\definecolor{dialinecolor}{rgb}{0.000000, 0.000000, 0.000000}
\pgfsetfillcolor{dialinecolor}
% was here!!!
\pgfsetarrowsstart{stealth}
\pgfsetarrowsend{stealth}
\definecolor{dialinecolor}{rgb}{0.000000, 0.000000, 0.000000}
\pgfsetstrokecolor{dialinecolor}
\draw (27.000000\du,15.500000\du)--(27.000000\du,17.500000\du);
}
% setfont left to latex
\definecolor{dialinecolor}{rgb}{0.000000, 0.000000, 0.000000}
\pgfsetstrokecolor{dialinecolor}
\node at (28.000000\du,16.721250\du){$q_3$};
\pgfsetlinewidth{0.050000\du}
\pgfsetdash{{0.300000\du}{0.300000\du}}{0\du}
\pgfsetdash{{0.300000\du}{0.300000\du}}{0\du}
\pgfsetbuttcap
{
\definecolor{dialinecolor}{rgb}{0.000000, 0.000000, 0.000000}
\pgfsetfillcolor{dialinecolor}
% was here!!!
\definecolor{dialinecolor}{rgb}{0.000000, 0.000000, 0.000000}
\pgfsetstrokecolor{dialinecolor}
\draw (15.000000\du,12.500000\du)--(27.500000\du,12.500000\du);
}
% setfont left to latex
\definecolor{dialinecolor}{rgb}{0.000000, 0.000000, 0.000000}
\pgfsetstrokecolor{dialinecolor}
\node at (14.000000\du,12.721250\du){$Q_r$};
\pgfsetlinewidth{0.150000\du}
\pgfsetdash{}{0pt}
\pgfsetdash{}{0pt}
\pgfsetbuttcap
{
\definecolor{dialinecolor}{rgb}{0.000000, 0.000000, 0.000000}
\pgfsetfillcolor{dialinecolor}
% was here!!!
\definecolor{dialinecolor}{rgb}{0.000000, 0.000000, 0.000000}
\pgfsetstrokecolor{dialinecolor}
\draw (14.960000\du,13.500000\du)--(18.070000\du,13.500000\du);
}
\pgfsetlinewidth{0.150000\du}
\pgfsetdash{}{0pt}
\pgfsetdash{}{0pt}
\pgfsetbuttcap
{
\definecolor{dialinecolor}{rgb}{0.000000, 0.000000, 0.000000}
\pgfsetfillcolor{dialinecolor}
% was here!!!
\definecolor{dialinecolor}{rgb}{0.000000, 0.000000, 0.000000}
\pgfsetstrokecolor{dialinecolor}
\draw (18.000000\du,13.430000\du)--(18.000000\du,16.070000\du);
}
\pgfsetlinewidth{0.150000\du}
\pgfsetdash{}{0pt}
\pgfsetdash{}{0pt}
\pgfsetbuttcap
{
\definecolor{dialinecolor}{rgb}{0.000000, 0.000000, 0.000000}
\pgfsetfillcolor{dialinecolor}
% was here!!!
\definecolor{dialinecolor}{rgb}{0.000000, 0.000000, 0.000000}
\pgfsetstrokecolor{dialinecolor}
\draw (17.930000\du,16.000000\du)--(19.570000\du,16.000000\du);
}
\pgfsetlinewidth{0.150000\du}
\pgfsetdash{}{0pt}
\pgfsetdash{}{0pt}
\pgfsetbuttcap
{
\definecolor{dialinecolor}{rgb}{0.000000, 0.000000, 0.000000}
\pgfsetfillcolor{dialinecolor}
% was here!!!
\definecolor{dialinecolor}{rgb}{0.000000, 0.000000, 0.000000}
\pgfsetstrokecolor{dialinecolor}
\draw (19.500000\du,15.930000\du)--(19.500000\du,17.570000\du);
}
\pgfsetlinewidth{0.150000\du}
\pgfsetdash{}{0pt}
\pgfsetdash{}{0pt}
\pgfsetbuttcap
{
\definecolor{dialinecolor}{rgb}{0.000000, 0.000000, 0.000000}
\pgfsetfillcolor{dialinecolor}
% was here!!!
\definecolor{dialinecolor}{rgb}{0.000000, 0.000000, 0.000000}
\pgfsetstrokecolor{dialinecolor}
\draw (19.430000\du,17.500000\du)--(26.070000\du,17.500000\du);
}
\pgfsetlinewidth{0.150000\du}
\pgfsetdash{}{0pt}
\pgfsetdash{}{0pt}
\pgfsetbuttcap
{
\definecolor{dialinecolor}{rgb}{0.000000, 0.000000, 0.000000}
\pgfsetfillcolor{dialinecolor}
% was here!!!
\definecolor{dialinecolor}{rgb}{0.000000, 0.000000, 0.000000}
\pgfsetstrokecolor{dialinecolor}
\draw (26.000000\du,15.430000\du)--(26.000000\du,17.570000\du);
}
\pgfsetlinewidth{0.150000\du}
\pgfsetdash{}{0pt}
\pgfsetdash{}{0pt}
\pgfsetbuttcap
{
\definecolor{dialinecolor}{rgb}{0.000000, 0.000000, 0.000000}
\pgfsetfillcolor{dialinecolor}
% was here!!!
\definecolor{dialinecolor}{rgb}{0.000000, 0.000000, 0.000000}
\pgfsetstrokecolor{dialinecolor}
\draw (25.930000\du,15.500000\du)--(27.570000\du,15.500000\du);
}
\pgfsetlinewidth{0.100000\du}
\pgfsetdash{}{0pt}
\pgfsetdash{}{0pt}
\pgfsetmiterjoin
\definecolor{dialinecolor}{rgb}{0.749020, 0.749020, 0.749020}
\pgfsetfillcolor{dialinecolor}
\fill (18.000000\du,21.500000\du)--(18.000000\du,22.000000\du)--(24.000000\du,22.000000\du)--(24.000000\du,21.500000\du)--cycle;
\definecolor{dialinecolor}{rgb}{0.000000, 0.000000, 0.000000}
\pgfsetstrokecolor{dialinecolor}
\draw (18.000000\du,21.500000\du)--(18.000000\du,22.000000\du)--(24.000000\du,22.000000\du)--(24.000000\du,21.500000\du)--cycle;
% setfont left to latex
\definecolor{dialinecolor}{rgb}{0.000000, 0.000000, 0.000000}
\pgfsetstrokecolor{dialinecolor}
\node at (21.000000\du,22.711906\du){$consumir(t_1, r, q_1)$};
\pgfsetlinewidth{0.100000\du}
\pgfsetdash{}{0pt}
\pgfsetdash{}{0pt}
\pgfsetmiterjoin
\definecolor{dialinecolor}{rgb}{0.749020, 0.749020, 0.749020}
\pgfsetfillcolor{dialinecolor}
\fill (22.000000\du,23.500000\du)--(22.000000\du,24.000000\du)--(26.000000\du,24.000000\du)--(26.000000\du,23.500000\du)--cycle;
\definecolor{dialinecolor}{rgb}{0.000000, 0.000000, 0.000000}
\pgfsetstrokecolor{dialinecolor}
\draw (22.000000\du,23.500000\du)--(22.000000\du,24.000000\du)--(26.000000\du,24.000000\du)--(26.000000\du,23.500000\du)--cycle;
% setfont left to latex
\definecolor{dialinecolor}{rgb}{0.000000, 0.000000, 0.000000}
\pgfsetstrokecolor{dialinecolor}
\node at (24.000000\du,24.711906\du){$produzir(t_3, r, q_3)$};
\pgfsetlinewidth{0.100000\du}
\pgfsetdash{}{0pt}
\pgfsetdash{}{0pt}
\pgfsetmiterjoin
\definecolor{dialinecolor}{rgb}{0.749020, 0.749020, 0.749020}
\pgfsetfillcolor{dialinecolor}
\fill (19.500000\du,25.500000\du)--(19.500000\du,26.000000\du)--(22.000000\du,26.000000\du)--(22.000000\du,25.500000\du)--cycle;
\definecolor{dialinecolor}{rgb}{0.000000, 0.000000, 0.000000}
\pgfsetstrokecolor{dialinecolor}
\draw (19.500000\du,25.500000\du)--(19.500000\du,26.000000\du)--(22.000000\du,26.000000\du)--(22.000000\du,25.500000\du)--cycle;
% setfont left to latex
\definecolor{dialinecolor}{rgb}{0.000000, 0.000000, 0.000000}
\pgfsetstrokecolor{dialinecolor}
\node at (20.750000\du,26.711906\du){$consumir(t_2, r, q_2)$};
\end{tikzpicture}

            \caption{Perfil característico de um recurso consumível.} \label{fig:recurso_consumivel}
        \end{figure}
        
        \emph{\color{red} (dar exemplos gráficos de algumas funções de consumo/produção: \textit{step}, \textit{pulse}, \textit{linear} e \textit{exponential}). }
        
    
    % 3a seção do capítulo 4
    \section{Tipos de Recurso} \label{sec:tipos_recurso}
    
        Recursos possuem um tipo, podendo ele ser: (1) contínuo, (2) discreto ou (3) unário. 
        
        Primeiramente, em \textit{recursos contínuos}, a capacidade total do recurso é definida por um número pertencente ao conjunto dos números reais estritamente positivos, enquanto sua quantidade corrente é uma representação numérica que pertence ao conjunto dos números reais não-negativos. Assim temos, 
        
        \begin{equation} \label{eq:recurso_continuo}
            z_r: t \in \mathbb{R}_+ \to z \in [0; Q_r] \subset \mathbb{R}_+ \mid Q_r \in \mathbb{R}_+^*
        \end{equation}
        
        Exemplificando \dots \emph{\color{red} (dar exemplo(s) de recursos reusáveis contínuos e recursos consumíveis contínuos. Falar do tipo de funções tbm: \textit{step}, \textit{pulse}, \textit{linear} e \textit{exponential})}
        
        \textit{Recursos discretos} possuem capacidade total definida por um número inteiro estritamente positivo, isto é, um número natural, e quantidade corrente por um número inteiro não-negativo, obtendo:
        
        \begin{equation} \label{eq:recurso_discreto}
            z_r: t \in \mathbb{R}_+ \to z \in [0; Q_r] \subset \mathbb{Z}_+ \mid Q_r \in \mathbb{N}
        \end{equation}
        
        Exemplificando \dots \emph{\color{red} (dar exemplo(s) de recursos reusáveis discretos e recursos consumíveis discretos. Falar do tipo de funções tbm: \textit{step}, \textit{pulse}, \textit{linear} e \textit{exponential})}
        
        E, finalmente, um \textit{recurso unário} sempre possui capacidade total igual à $1$ e sua quantidade corrente pode assumir os valores $0$. Com isso, podemos concluir que a quantidade corrente do recurso informa sua disponibilidade ao longo do tempo: quando $0$, o recurso se encontra indisponível; e, quando $1$, o recurso está disponível. Em outras palavras,
        
        \begin{equation} \label{eq:recurso_unario}
            z_r: t \in \mathbb{R}_+ \to z \in \{0; 1\} 
            \quad\text{e}\quad
            Q_r = 1
        \end{equation}
        
        Exemplificando \dots \emph{\color{red} (dar exemplo(s) de recursos reusáveis unários e recursos consumíveis unários. Falar do tipo de funções tbm: \textit{step}, \textit{pulse}, \textit{linear} e \textit{exponential})}
        
        \dots \emph{\color{red} (dar de uma aplicação com várias tarefas que utilizam recursos de classes e tipos variados.)}
