\chapter[Recursos]{Recursos} \label{cap:recursos}

    Recursos são altamente utilizados, compartilhados, produzidos e consumidos em aplicações industriais e de robótica. Para que os robôs sejam capazes de executar uma dada tarefa, normalmente é requerido que eles possuam um conjunto de recursos, sendo especificado a quantidade necessária de cada um deles. Por exemplo, para que um robô móvel possa adquirir e processar uma imagem digital, como parte de uma dada tarefa, é preciso que esse possua uma câmera digital e, dependendo da taxa de processamento desejada, um processador dedicado especialmente para a renderização de gráficos em tempo real, isto é, uma GPU (\textit{Graphics Processing Unit}).
    
    Seja uma linha de montagem para a produção em série de uma determinado produto, composta por grupo de robôs heterogêneos e, também, por máquinas automáticas. Os processos são realizados pelas máquinas de forma sequencial. E os robôs são responsáveis por transportar matéria-prima e produto semi-acabado de um processo a outro. Além disso, eles devem abastecer os suprimentos das máquinas quando notificados. \emph{\color{red} As máquinas possuem sinalizadores que identificam em qual estado elas se encontram: (1) \textit{aguardando}, modo em que a máquina está ociosa, esperando um novo produto semi-acabado entrar para processar; (2) \textit{preparando}, modo em que a máquina se configura para processar o produto semi-acabado recebido; (3) \textit{\textbf{processando peça}}, modo em que a máquina está trabalhando o produto semi-acabado; (4) \textit{\textbf{peça processada}}, produto já recebeu todo trabalho necessário nesta etapa; (5) \textit{danificada}, ; (6) \textit{em manutenção}, ; e (7) \textit{em abastecimento}, .}
    
    \section{Tarefas} \label{sec:tarefa}
        
        Uma tarefa $a$ pode ser especificada pelos seus requerimentos de recurso, pelo seu tempo de início $s(a)$, pelo seu tempo de término $e(a)$, bem como, sua duração $d(a)$. Em aplicações reais, o tempo de início, de término e a duração de uma tarefa são estimações apenas. Assim, $s(a)$ e $e(a)$ são especificados como variáveis estocásticas que possuem uma alta probabilidade de ocorrer dentro dos intervalos: $s(a) \in [s_{min}(a); s_{max}(a)]$ e $e(a) \in [e_{min}(a); e_{max}(a)]$. Podemos ainda definir esses dois parâmetros como distribuições normais: 
        %
        $$
            \begin{aligned}
                s(a) &\sim \mathcal{N}(\mu_{s(a)}, \sigma_{s(a)}^2) \\
                e(a) &\sim \mathcal{N}(\mu_{e(a)}, \sigma_{e(a)}^2)
            \end{aligned}
        $$
        %
        Neste caso, com uma probabilidade de $99,7\%$, teremos: 
        %
        $$
            \begin{aligned}
                s_{min}(a) &\approx \mu_{s(a)} - 3\sigma_{s(a)} \\
                s_{max}(a) &\approx \mu_{s(a)} + 3\sigma_{s(a)} \\
                e_{min}(a) &\approx \mu_{e(a)} - 3\sigma_{e(a)} \\
                e_{max}(a) &\approx \mu_{e(a)} + 3\sigma_{e(a)}
            \end{aligned}
        $$
        
        Tarefas podem ainda ser \textit{preemptivas} ou \textit{não-preemptivas}. Tarefas não-preemptivas devem ser executadas sem interrupções, de modo que, $d(a) = e(a) - s(a)$. Entretanto, podem ocorrer interrupções durante a execução de tarefas preemptivas. Neste caso, os recursos utilizados por $a$ são liberados para que outras tarefas possam usá-los. O cálculo da duração de tarefas preemptivas é dado por:
        %
        $$
            d(a) = \sum\limits_{i=1}^k d_i \leq e(a) - s(a)
        $$
        %
        em que $d_i(a)$ é a duração de cada interrupção. \emph{\color{red}Serão consideradas, no restante deste trabalho, apenas tarefas não-preemptivas.}
        
        Contudo, o foco deste capítulo e deste trabalho está nos requerimentos de recurso que as tarefas da missão fazem aos robôs do sistema. Conforme dito anteriormente, ao especificar uma tarefa é necessário informar quais são os recursos necessários para sua execução e completude. Assim, nos próximos parágrafos deste capítulo, serão definidos as classes e os tipos de recursos existentes.

    Existem duas classes de recursos, basicamente: aqueles que podem ser usados novamente, os quais são denominados \textit{recursos reusáveis}; e aqueles que são descartáveis, denominados \textit{recursos consumíveis} \cite{ref:ghallab2004automated}. Cada classe de recurso é caracterizada por um \textit{perfil}. O perfil de um recurso $r$ é uma função do tempo, $z_r(t)$, que define sua quantidade instantânea. A seguir, cada classe de recurso será definida, bem como, será detalhado o perfil de cada uma delas.
    
    % 1a seção do capítulo 4
    \section{Recursos Reusáveis} \label{sec:recurso_reusavel}
    
        Um recurso reusável é ``emprestado'' por uma tarefa durante sua execução. E quando ela é interrompida ou finalizada, esse recurso é liberado sem alterações. Assim, um recurso reusável $r$ tem uma capacidade total $Q_r$ e uma quantidade corrente $z_r(t) \in [0; Q_r]$. 
        
        Seja uma tarefa $a$ que requer durante a sua execução uma quantidade $q$ do recurso $r$. Ao ser iniciada, no tempo $s(a)$, a quantidade corrente de $r$ é diminuída em um montante $q$. Entretanto, quando $a$ é finalizada, no tempo $e(a)$, a quantidade corrente de $r$ é aumentada pelo mesmo montante, $q$. Portanto, um recurso reusável possui um perfil característico conforme mostra a Figura \ref{fig:recurso_reusavel}.
        
        \begin{figure}[htb]
            \centering
            % Graphic for TeX using PGF
% Title: ../figures/reusable_resource.dia
% Creator: Dia v0.97.2
% CreationDate: Tue Oct 17 10:11:20 2017
% For: adrianohrl
% \usepackage{tikz}
% The following commands are not supported in PSTricks at present
% We define them conditionally, so when they are implemented,
% this pgf file will use them.
\ifx\du\undefined
  \newlength{\du}
\fi
\setlength{\du}{15\unitlength}
\begin{tikzpicture}
\pgftransformxscale{1.000000}
\pgftransformyscale{-1.000000}
\definecolor{dialinecolor}{rgb}{0.000000, 0.000000, 0.000000}
\pgfsetstrokecolor{dialinecolor}
\definecolor{dialinecolor}{rgb}{1.000000, 1.000000, 1.000000}
\pgfsetfillcolor{dialinecolor}
\pgfsetlinewidth{0.050000\du}
\pgfsetdash{{1.000000\du}{1.000000\du}}{0\du}
\pgfsetdash{{0.300000\du}{0.300000\du}}{0\du}
\pgfsetbuttcap
{
\definecolor{dialinecolor}{rgb}{0.000000, 0.000000, 0.000000}
\pgfsetfillcolor{dialinecolor}
% was here!!!
\definecolor{dialinecolor}{rgb}{0.000000, 0.000000, 0.000000}
\pgfsetstrokecolor{dialinecolor}
\draw (27.500000\du,17.500000\du)--(29.000000\du,17.500000\du);
}
\pgfsetlinewidth{0.050000\du}
\pgfsetdash{{0.300000\du}{0.300000\du}}{0\du}
\pgfsetdash{{0.300000\du}{0.300000\du}}{0\du}
\pgfsetbuttcap
{
\definecolor{dialinecolor}{rgb}{0.000000, 0.000000, 0.000000}
\pgfsetfillcolor{dialinecolor}
% was here!!!
\pgfsetarrowsstart{stealth}
\pgfsetarrowsend{stealth}
\definecolor{dialinecolor}{rgb}{0.000000, 0.000000, 0.000000}
\pgfsetstrokecolor{dialinecolor}
\draw (28.000000\du,16.000000\du)--(28.000000\du,17.500000\du);
}
% setfont left to latex
\definecolor{dialinecolor}{rgb}{0.000000, 0.000000, 0.000000}
\pgfsetstrokecolor{dialinecolor}
\node at (27.000000\du,16.971250\du){$q_2$};
\pgfsetlinewidth{0.050000\du}
\pgfsetdash{{0.300000\du}{0.300000\du}}{0\du}
\pgfsetdash{{0.300000\du}{0.300000\du}}{0\du}
\pgfsetbuttcap
{
\definecolor{dialinecolor}{rgb}{0.000000, 0.000000, 0.000000}
\pgfsetfillcolor{dialinecolor}
% was here!!!
\definecolor{dialinecolor}{rgb}{0.000000, 0.000000, 0.000000}
\pgfsetstrokecolor{dialinecolor}
\draw (26.000000\du,16.000000\du)--(27.500000\du,16.000000\du);
}
\pgfsetlinewidth{0.050000\du}
\pgfsetdash{{0.300000\du}{0.300000\du}}{0\du}
\pgfsetdash{{0.300000\du}{0.300000\du}}{0\du}
\pgfsetbuttcap
{
\definecolor{dialinecolor}{rgb}{0.000000, 0.000000, 0.000000}
\pgfsetfillcolor{dialinecolor}
% was here!!!
\pgfsetarrowsstart{stealth}
\pgfsetarrowsend{stealth}
\definecolor{dialinecolor}{rgb}{0.000000, 0.000000, 0.000000}
\pgfsetstrokecolor{dialinecolor}
\draw (26.500000\du,12.500000\du)--(26.500000\du,16.000000\du);
}
% setfont left to latex
\definecolor{dialinecolor}{rgb}{0.000000, 0.000000, 0.000000}
\pgfsetstrokecolor{dialinecolor}
\node at (25.500000\du,14.471250\du){$q_1$};
\pgfsetlinewidth{0.050000\du}
\pgfsetdash{{0.300000\du}{0.300000\du}}{0\du}
\pgfsetdash{{0.300000\du}{0.300000\du}}{0\du}
\pgfsetbuttcap
{
\definecolor{dialinecolor}{rgb}{0.000000, 0.000000, 0.000000}
\pgfsetfillcolor{dialinecolor}
% was here!!!
\definecolor{dialinecolor}{rgb}{0.000000, 0.000000, 0.000000}
\pgfsetstrokecolor{dialinecolor}
\draw (24.000000\du,12.500000\du)--(36.500000\du,12.500000\du);
}
% setfont left to latex
\definecolor{dialinecolor}{rgb}{0.000000, 0.000000, 0.000000}
\pgfsetstrokecolor{dialinecolor}
\node at (23.000000\du,12.721250\du){$Q_r$};
\pgfsetlinewidth{0.150000\du}
\pgfsetdash{}{0pt}
\pgfsetdash{}{0pt}
\pgfsetbuttcap
{
\definecolor{dialinecolor}{rgb}{0.000000, 0.000000, 0.000000}
\pgfsetfillcolor{dialinecolor}
% was here!!!
\definecolor{dialinecolor}{rgb}{0.000000, 0.000000, 0.000000}
\pgfsetstrokecolor{dialinecolor}
\draw (23.960000\du,12.500000\du)--(27.570000\du,12.500000\du);
}
\pgfsetlinewidth{0.150000\du}
\pgfsetdash{}{0pt}
\pgfsetdash{}{0pt}
\pgfsetbuttcap
{
\definecolor{dialinecolor}{rgb}{0.000000, 0.000000, 0.000000}
\pgfsetfillcolor{dialinecolor}
% was here!!!
\definecolor{dialinecolor}{rgb}{0.000000, 0.000000, 0.000000}
\pgfsetstrokecolor{dialinecolor}
\draw (27.500000\du,12.430000\du)--(27.500000\du,16.070000\du);
}
\pgfsetlinewidth{0.150000\du}
\pgfsetdash{}{0pt}
\pgfsetdash{}{0pt}
\pgfsetbuttcap
{
\definecolor{dialinecolor}{rgb}{0.000000, 0.000000, 0.000000}
\pgfsetfillcolor{dialinecolor}
% was here!!!
\definecolor{dialinecolor}{rgb}{0.000000, 0.000000, 0.000000}
\pgfsetstrokecolor{dialinecolor}
\draw (27.430000\du,16.000000\du)--(29.070000\du,16.000000\du);
}
\pgfsetlinewidth{0.150000\du}
\pgfsetdash{}{0pt}
\pgfsetdash{}{0pt}
\pgfsetbuttcap
{
\definecolor{dialinecolor}{rgb}{0.000000, 0.000000, 0.000000}
\pgfsetfillcolor{dialinecolor}
% was here!!!
\definecolor{dialinecolor}{rgb}{0.000000, 0.000000, 0.000000}
\pgfsetstrokecolor{dialinecolor}
\draw (29.000000\du,15.930000\du)--(29.000000\du,17.570000\du);
}
\pgfsetlinewidth{0.150000\du}
\pgfsetdash{}{0pt}
\pgfsetdash{}{0pt}
\pgfsetbuttcap
{
\definecolor{dialinecolor}{rgb}{0.000000, 0.000000, 0.000000}
\pgfsetfillcolor{dialinecolor}
% was here!!!
\definecolor{dialinecolor}{rgb}{0.000000, 0.000000, 0.000000}
\pgfsetstrokecolor{dialinecolor}
\draw (28.930000\du,17.500000\du)--(32.570000\du,17.500000\du);
}
\pgfsetlinewidth{0.150000\du}
\pgfsetdash{}{0pt}
\pgfsetdash{}{0pt}
\pgfsetbuttcap
{
\definecolor{dialinecolor}{rgb}{0.000000, 0.000000, 0.000000}
\pgfsetfillcolor{dialinecolor}
% was here!!!
\definecolor{dialinecolor}{rgb}{0.000000, 0.000000, 0.000000}
\pgfsetstrokecolor{dialinecolor}
\draw (32.500000\du,13.930000\du)--(32.500000\du,17.570000\du);
}
\pgfsetlinewidth{0.150000\du}
\pgfsetdash{}{0pt}
\pgfsetdash{}{0pt}
\pgfsetbuttcap
{
\definecolor{dialinecolor}{rgb}{0.000000, 0.000000, 0.000000}
\pgfsetfillcolor{dialinecolor}
% was here!!!
\definecolor{dialinecolor}{rgb}{0.000000, 0.000000, 0.000000}
\pgfsetstrokecolor{dialinecolor}
\draw (32.430000\du,14.000000\du)--(33.570000\du,14.000000\du);
}
\pgfsetlinewidth{0.150000\du}
\pgfsetdash{}{0pt}
\pgfsetdash{}{0pt}
\pgfsetbuttcap
{
\definecolor{dialinecolor}{rgb}{0.000000, 0.000000, 0.000000}
\pgfsetfillcolor{dialinecolor}
% was here!!!
\definecolor{dialinecolor}{rgb}{0.000000, 0.000000, 0.000000}
\pgfsetstrokecolor{dialinecolor}
\draw (33.500000\du,12.430000\du)--(33.500000\du,14.070000\du);
}
\pgfsetlinewidth{0.150000\du}
\pgfsetdash{}{0pt}
\pgfsetdash{}{0pt}
\pgfsetbuttcap
{
\definecolor{dialinecolor}{rgb}{0.000000, 0.000000, 0.000000}
\pgfsetfillcolor{dialinecolor}
% was here!!!
\definecolor{dialinecolor}{rgb}{0.000000, 0.000000, 0.000000}
\pgfsetstrokecolor{dialinecolor}
\draw (33.430000\du,12.500000\du)--(36.570000\du,12.500000\du);
}
\pgfsetlinewidth{0.100000\du}
\pgfsetdash{}{0pt}
\pgfsetdash{}{0pt}
\pgfsetmiterjoin
\definecolor{dialinecolor}{rgb}{0.749020, 0.749020, 0.749020}
\pgfsetfillcolor{dialinecolor}
\fill (27.500000\du,21.500000\du)--(27.500000\du,22.000000\du)--(32.500000\du,22.000000\du)--(32.500000\du,21.500000\du)--cycle;
\definecolor{dialinecolor}{rgb}{0.000000, 0.000000, 0.000000}
\pgfsetstrokecolor{dialinecolor}
\draw (27.500000\du,21.500000\du)--(27.500000\du,22.000000\du)--(32.500000\du,22.000000\du)--(32.500000\du,21.500000\du)--cycle;
% setfont left to latex
\definecolor{dialinecolor}{rgb}{0.000000, 0.000000, 0.000000}
\pgfsetstrokecolor{dialinecolor}
\node at (30.000000\du,22.721250\du){$requerer(a_1, r, q_1)$};
\pgfsetlinewidth{0.100000\du}
\pgfsetdash{}{0pt}
\pgfsetdash{}{0pt}
\pgfsetmiterjoin
\definecolor{dialinecolor}{rgb}{0.749020, 0.749020, 0.749020}
\pgfsetfillcolor{dialinecolor}
\fill (29.000000\du,23.500000\du)--(29.000000\du,24.000000\du)--(33.500000\du,24.000000\du)--(33.500000\du,23.500000\du)--cycle;
\definecolor{dialinecolor}{rgb}{0.000000, 0.000000, 0.000000}
\pgfsetstrokecolor{dialinecolor}
\draw (29.000000\du,23.500000\du)--(29.000000\du,24.000000\du)--(33.500000\du,24.000000\du)--(33.500000\du,23.500000\du)--cycle;
% setfont left to latex
\definecolor{dialinecolor}{rgb}{0.000000, 0.000000, 0.000000}
\pgfsetstrokecolor{dialinecolor}
\node at (31.250000\du,24.721250\du){$requerer(a_2, r, q_2)$};
\pgfsetlinewidth{0.100000\du}
\pgfsetdash{}{0pt}
\pgfsetdash{}{0pt}
\pgfsetbuttcap
{
\definecolor{dialinecolor}{rgb}{0.000000, 0.000000, 0.000000}
\pgfsetfillcolor{dialinecolor}
% was here!!!
\pgfsetarrowsend{stealth}
\definecolor{dialinecolor}{rgb}{0.000000, 0.000000, 0.000000}
\pgfsetstrokecolor{dialinecolor}
\draw (24.000000\du,21.000000\du)--(24.000000\du,11.000000\du);
}
\pgfsetlinewidth{0.100000\du}
\pgfsetdash{}{0pt}
\pgfsetdash{}{0pt}
\pgfsetbuttcap
{
\definecolor{dialinecolor}{rgb}{0.000000, 0.000000, 0.000000}
\pgfsetfillcolor{dialinecolor}
% was here!!!
\pgfsetarrowsend{stealth}
\definecolor{dialinecolor}{rgb}{0.000000, 0.000000, 0.000000}
\pgfsetstrokecolor{dialinecolor}
\draw (23.000000\du,20.000000\du)--(38.000000\du,20.000000\du);
}
% setfont left to latex
\definecolor{dialinecolor}{rgb}{0.000000, 0.000000, 0.000000}
\pgfsetstrokecolor{dialinecolor}
\node at (24.000000\du,10.221250\du){$z_r(t)$};
% setfont left to latex
\definecolor{dialinecolor}{rgb}{0.000000, 0.000000, 0.000000}
\pgfsetstrokecolor{dialinecolor}
\node at (39.000000\du,20.221250\du){$t$};
\end{tikzpicture}

            \caption{Perfil característico de um recurso reusável.} \label{fig:recurso_reusavel}
        \end{figure}
        
        Verifica-se, na Figura \ref{fig:recurso_reusavel}, que o recurso $r$ possui inicialmente uma quantidade equivalente à sua capacidade total, isto é, $z_r(0) = Q_r$. Isso acontece pois nenhuma tarefa fez requisição do recurso $r$. Entretanto, quando a tarefa $a_1$ é iniciada, a quantidade do recurso $r$, em $s(a_1)$, passa a ser $z_r(s(a_1)) = Q_r - q_1$. Em $s(a_2)$, a tarefa $a_2$ é iniciada. Com isso, a quantidade de $r$ passa a valer $z_r(s(a_2)) = Q_r - q_1 - q_2$, pois a tarefa $a_1$ ainda está em execução, utilizando uma quantia $q_1$ de $r$. Quando $a_1$ é finalizada, no instante $e(a_1)$, essa deixa de utilizar o recurso $r$, o qual passa a ter uma quantidade $z_r(e(a_1)) = Q_r - q_2$. Finalmente, ao término de $a_2$, em $e(a_2)$, todas as requisições de $r$ se encerram, fazendo com que sua quantidade volte ao valor de sua capacidade, de modo que, $z_r(e(a_2)) = Q_r$. É importante lembrar que, após cada variação de quantidade, a quantidade do recurso reusável $r$ se mantém constante até que um novo evento ocorra, ou seja, até que uma nova requisição seja iniciada ou encerrada.
    
    % 2a seção do capítulo 4
    \section{Recursos Consumíveis} \label{sec:recurso_consumivel}
    
        Um recurso consumível pode ser produzido ou consumido durante a execução de uma tarefa. Esta classe de recurso pode ser modelada como um reservatório de capacidade máxima limitada $Q_r$ e nível (quantidade) corrente $z_r(t) \in [0; Q_r]$.
        
        Seja, pois, uma tarefa $a$ que produz uma quantidade $q$ do recurso $r$ durante a sua execução. Quando iniciada, em $s(a)$, aumenta um montante $q$ do seu nível $z_r(t)$ ao longo do tempo. Essa produção é modelada por uma função dependente do tempo, crescente no intervalo temporal $[s(a); e(a)]$. 
        
        Seja, agora, uma tarefa $a$ que consome uma quantidade $q$ do recurso $r$ durante a sua execução. Quando iniciada, no instante $s(a)$, reduz um montante $q$ do seu nível $z_r(t)$ ao longo do tempo. Essa redução/consumo é modelada como uma função do tempo, decrescente no intervalo de tempo $[s(a); e(a)]$.  Portanto, um recurso consumível possui um perfil característico conforme mostrado na Figura \ref{fig:recurso_consumivel}.
        
        \begin{figure}[htb]
            \centering
            % Graphic for TeX using PGF
% Title: ../figures/consumable_resource.dia
% Creator: Dia v0.97.2
% CreationDate: Tue Oct 17 10:11:22 2017
% For: adrianohrl
% \usepackage{tikz}
% The following commands are not supported in PSTricks at present
% We define them conditionally, so when they are implemented,
% this pgf file will use them.
\ifx\du\undefined
  \newlength{\du}
\fi
\setlength{\du}{15\unitlength}
\begin{tikzpicture}
\pgftransformxscale{1.000000}
\pgftransformyscale{-1.000000}
\definecolor{dialinecolor}{rgb}{0.000000, 0.000000, 0.000000}
\pgfsetstrokecolor{dialinecolor}
\definecolor{dialinecolor}{rgb}{1.000000, 1.000000, 1.000000}
\pgfsetfillcolor{dialinecolor}
\pgfsetlinewidth{0.050000\du}
\pgfsetdash{{1.000000\du}{1.000000\du}}{0\du}
\pgfsetdash{{0.300000\du}{0.300000\du}}{0\du}
\pgfsetbuttcap
{
\definecolor{dialinecolor}{rgb}{0.000000, 0.000000, 0.000000}
\pgfsetfillcolor{dialinecolor}
% was here!!!
\definecolor{dialinecolor}{rgb}{0.000000, 0.000000, 0.000000}
\pgfsetstrokecolor{dialinecolor}
\draw (16.500000\du,16.000000\du)--(18.000000\du,16.000000\du);
}
\pgfsetlinewidth{0.050000\du}
\pgfsetdash{{0.300000\du}{0.300000\du}}{0\du}
\pgfsetdash{{0.300000\du}{0.300000\du}}{0\du}
\pgfsetbuttcap
{
\definecolor{dialinecolor}{rgb}{0.000000, 0.000000, 0.000000}
\pgfsetfillcolor{dialinecolor}
% was here!!!
\pgfsetarrowsstart{stealth}
\pgfsetarrowsend{stealth}
\definecolor{dialinecolor}{rgb}{0.000000, 0.000000, 0.000000}
\pgfsetstrokecolor{dialinecolor}
\draw (17.000000\du,13.500000\du)--(17.000000\du,16.000000\du);
}
% setfont left to latex
\definecolor{dialinecolor}{rgb}{0.000000, 0.000000, 0.000000}
\pgfsetstrokecolor{dialinecolor}
\node at (16.000000\du,14.971250\du){$q_1$};
\pgfsetlinewidth{0.050000\du}
\pgfsetdash{{0.300000\du}{0.300000\du}}{0\du}
\pgfsetdash{{0.300000\du}{0.300000\du}}{0\du}
\pgfsetbuttcap
{
\definecolor{dialinecolor}{rgb}{0.000000, 0.000000, 0.000000}
\pgfsetfillcolor{dialinecolor}
% was here!!!
\definecolor{dialinecolor}{rgb}{0.000000, 0.000000, 0.000000}
\pgfsetstrokecolor{dialinecolor}
\draw (18.000000\du,17.500000\du)--(19.500000\du,17.500000\du);
}
\pgfsetlinewidth{0.050000\du}
\pgfsetdash{{0.300000\du}{0.300000\du}}{0\du}
\pgfsetdash{{0.300000\du}{0.300000\du}}{0\du}
\pgfsetbuttcap
{
\definecolor{dialinecolor}{rgb}{0.000000, 0.000000, 0.000000}
\pgfsetfillcolor{dialinecolor}
% was here!!!
\pgfsetarrowsstart{stealth}
\pgfsetarrowsend{stealth}
\definecolor{dialinecolor}{rgb}{0.000000, 0.000000, 0.000000}
\pgfsetstrokecolor{dialinecolor}
\draw (18.500000\du,16.000000\du)--(18.500000\du,17.500000\du);
}
% setfont left to latex
\definecolor{dialinecolor}{rgb}{0.000000, 0.000000, 0.000000}
\pgfsetstrokecolor{dialinecolor}
\node at (17.500000\du,16.971250\du){$q_2$};
\pgfsetlinewidth{0.050000\du}
\pgfsetdash{{0.300000\du}{0.300000\du}}{0\du}
\pgfsetdash{{0.300000\du}{0.300000\du}}{0\du}
\pgfsetbuttcap
{
\definecolor{dialinecolor}{rgb}{0.000000, 0.000000, 0.000000}
\pgfsetfillcolor{dialinecolor}
% was here!!!
\definecolor{dialinecolor}{rgb}{0.000000, 0.000000, 0.000000}
\pgfsetstrokecolor{dialinecolor}
\draw (26.000000\du,17.500000\du)--(27.500000\du,17.500000\du);
}
\pgfsetlinewidth{0.050000\du}
\pgfsetdash{{0.300000\du}{0.300000\du}}{0\du}
\pgfsetdash{{0.300000\du}{0.300000\du}}{0\du}
\pgfsetbuttcap
{
\definecolor{dialinecolor}{rgb}{0.000000, 0.000000, 0.000000}
\pgfsetfillcolor{dialinecolor}
% was here!!!
\pgfsetarrowsstart{stealth}
\pgfsetarrowsend{stealth}
\definecolor{dialinecolor}{rgb}{0.000000, 0.000000, 0.000000}
\pgfsetstrokecolor{dialinecolor}
\draw (27.000000\du,15.500000\du)--(27.000000\du,17.500000\du);
}
% setfont left to latex
\definecolor{dialinecolor}{rgb}{0.000000, 0.000000, 0.000000}
\pgfsetstrokecolor{dialinecolor}
\node at (28.000000\du,16.721250\du){$q_3$};
\pgfsetlinewidth{0.050000\du}
\pgfsetdash{{0.300000\du}{0.300000\du}}{0\du}
\pgfsetdash{{0.300000\du}{0.300000\du}}{0\du}
\pgfsetbuttcap
{
\definecolor{dialinecolor}{rgb}{0.000000, 0.000000, 0.000000}
\pgfsetfillcolor{dialinecolor}
% was here!!!
\definecolor{dialinecolor}{rgb}{0.000000, 0.000000, 0.000000}
\pgfsetstrokecolor{dialinecolor}
\draw (15.000000\du,12.500000\du)--(27.500000\du,12.500000\du);
}
% setfont left to latex
\definecolor{dialinecolor}{rgb}{0.000000, 0.000000, 0.000000}
\pgfsetstrokecolor{dialinecolor}
\node at (14.000000\du,12.721250\du){$Q_r$};
\pgfsetlinewidth{0.150000\du}
\pgfsetdash{}{0pt}
\pgfsetdash{}{0pt}
\pgfsetbuttcap
{
\definecolor{dialinecolor}{rgb}{0.000000, 0.000000, 0.000000}
\pgfsetfillcolor{dialinecolor}
% was here!!!
\definecolor{dialinecolor}{rgb}{0.000000, 0.000000, 0.000000}
\pgfsetstrokecolor{dialinecolor}
\draw (14.960000\du,13.500000\du)--(18.070000\du,13.500000\du);
}
\pgfsetlinewidth{0.150000\du}
\pgfsetdash{}{0pt}
\pgfsetdash{}{0pt}
\pgfsetbuttcap
{
\definecolor{dialinecolor}{rgb}{0.000000, 0.000000, 0.000000}
\pgfsetfillcolor{dialinecolor}
% was here!!!
\definecolor{dialinecolor}{rgb}{0.000000, 0.000000, 0.000000}
\pgfsetstrokecolor{dialinecolor}
\draw (18.000000\du,13.430000\du)--(18.000000\du,16.070000\du);
}
\pgfsetlinewidth{0.150000\du}
\pgfsetdash{}{0pt}
\pgfsetdash{}{0pt}
\pgfsetbuttcap
{
\definecolor{dialinecolor}{rgb}{0.000000, 0.000000, 0.000000}
\pgfsetfillcolor{dialinecolor}
% was here!!!
\definecolor{dialinecolor}{rgb}{0.000000, 0.000000, 0.000000}
\pgfsetstrokecolor{dialinecolor}
\draw (17.930000\du,16.000000\du)--(19.570000\du,16.000000\du);
}
\pgfsetlinewidth{0.150000\du}
\pgfsetdash{}{0pt}
\pgfsetdash{}{0pt}
\pgfsetbuttcap
{
\definecolor{dialinecolor}{rgb}{0.000000, 0.000000, 0.000000}
\pgfsetfillcolor{dialinecolor}
% was here!!!
\definecolor{dialinecolor}{rgb}{0.000000, 0.000000, 0.000000}
\pgfsetstrokecolor{dialinecolor}
\draw (19.500000\du,15.930000\du)--(19.500000\du,17.570000\du);
}
\pgfsetlinewidth{0.150000\du}
\pgfsetdash{}{0pt}
\pgfsetdash{}{0pt}
\pgfsetbuttcap
{
\definecolor{dialinecolor}{rgb}{0.000000, 0.000000, 0.000000}
\pgfsetfillcolor{dialinecolor}
% was here!!!
\definecolor{dialinecolor}{rgb}{0.000000, 0.000000, 0.000000}
\pgfsetstrokecolor{dialinecolor}
\draw (19.430000\du,17.500000\du)--(26.070000\du,17.500000\du);
}
\pgfsetlinewidth{0.150000\du}
\pgfsetdash{}{0pt}
\pgfsetdash{}{0pt}
\pgfsetbuttcap
{
\definecolor{dialinecolor}{rgb}{0.000000, 0.000000, 0.000000}
\pgfsetfillcolor{dialinecolor}
% was here!!!
\definecolor{dialinecolor}{rgb}{0.000000, 0.000000, 0.000000}
\pgfsetstrokecolor{dialinecolor}
\draw (26.000000\du,15.430000\du)--(26.000000\du,17.570000\du);
}
\pgfsetlinewidth{0.150000\du}
\pgfsetdash{}{0pt}
\pgfsetdash{}{0pt}
\pgfsetbuttcap
{
\definecolor{dialinecolor}{rgb}{0.000000, 0.000000, 0.000000}
\pgfsetfillcolor{dialinecolor}
% was here!!!
\definecolor{dialinecolor}{rgb}{0.000000, 0.000000, 0.000000}
\pgfsetstrokecolor{dialinecolor}
\draw (25.930000\du,15.500000\du)--(27.570000\du,15.500000\du);
}
\pgfsetlinewidth{0.100000\du}
\pgfsetdash{}{0pt}
\pgfsetdash{}{0pt}
\pgfsetmiterjoin
\definecolor{dialinecolor}{rgb}{0.749020, 0.749020, 0.749020}
\pgfsetfillcolor{dialinecolor}
\fill (18.000000\du,21.500000\du)--(18.000000\du,22.000000\du)--(24.000000\du,22.000000\du)--(24.000000\du,21.500000\du)--cycle;
\definecolor{dialinecolor}{rgb}{0.000000, 0.000000, 0.000000}
\pgfsetstrokecolor{dialinecolor}
\draw (18.000000\du,21.500000\du)--(18.000000\du,22.000000\du)--(24.000000\du,22.000000\du)--(24.000000\du,21.500000\du)--cycle;
% setfont left to latex
\definecolor{dialinecolor}{rgb}{0.000000, 0.000000, 0.000000}
\pgfsetstrokecolor{dialinecolor}
\node at (21.000000\du,22.721250\du){$consumir(a_1, r, q_1)$};
\pgfsetlinewidth{0.100000\du}
\pgfsetdash{}{0pt}
\pgfsetdash{}{0pt}
\pgfsetmiterjoin
\definecolor{dialinecolor}{rgb}{0.749020, 0.749020, 0.749020}
\pgfsetfillcolor{dialinecolor}
\fill (22.000000\du,23.500000\du)--(22.000000\du,24.000000\du)--(26.000000\du,24.000000\du)--(26.000000\du,23.500000\du)--cycle;
\definecolor{dialinecolor}{rgb}{0.000000, 0.000000, 0.000000}
\pgfsetstrokecolor{dialinecolor}
\draw (22.000000\du,23.500000\du)--(22.000000\du,24.000000\du)--(26.000000\du,24.000000\du)--(26.000000\du,23.500000\du)--cycle;
% setfont left to latex
\definecolor{dialinecolor}{rgb}{0.000000, 0.000000, 0.000000}
\pgfsetstrokecolor{dialinecolor}
\node at (24.000000\du,24.721250\du){$produzir(a_3, r, q_3)$};
\pgfsetlinewidth{0.100000\du}
\pgfsetdash{}{0pt}
\pgfsetdash{}{0pt}
\pgfsetmiterjoin
\definecolor{dialinecolor}{rgb}{0.749020, 0.749020, 0.749020}
\pgfsetfillcolor{dialinecolor}
\fill (19.500000\du,25.500000\du)--(19.500000\du,26.000000\du)--(22.000000\du,26.000000\du)--(22.000000\du,25.500000\du)--cycle;
\definecolor{dialinecolor}{rgb}{0.000000, 0.000000, 0.000000}
\pgfsetstrokecolor{dialinecolor}
\draw (19.500000\du,25.500000\du)--(19.500000\du,26.000000\du)--(22.000000\du,26.000000\du)--(22.000000\du,25.500000\du)--cycle;
% setfont left to latex
\definecolor{dialinecolor}{rgb}{0.000000, 0.000000, 0.000000}
\pgfsetstrokecolor{dialinecolor}
\node at (20.750000\du,26.721250\du){$consumir(a_2, r, q_2)$};
\pgfsetlinewidth{0.100000\du}
\pgfsetdash{}{0pt}
\pgfsetdash{}{0pt}
\pgfsetbuttcap
{
\definecolor{dialinecolor}{rgb}{0.000000, 0.000000, 0.000000}
\pgfsetfillcolor{dialinecolor}
% was here!!!
\pgfsetarrowsend{stealth}
\definecolor{dialinecolor}{rgb}{0.000000, 0.000000, 0.000000}
\pgfsetstrokecolor{dialinecolor}
\draw (15.000000\du,21.000000\du)--(15.000000\du,11.000000\du);
}
% setfont left to latex
\definecolor{dialinecolor}{rgb}{0.000000, 0.000000, 0.000000}
\pgfsetstrokecolor{dialinecolor}
\node at (15.000000\du,10.221250\du){$z_r(t)$};
\pgfsetlinewidth{0.100000\du}
\pgfsetdash{}{0pt}
\pgfsetdash{}{0pt}
\pgfsetbuttcap
{
\definecolor{dialinecolor}{rgb}{0.000000, 0.000000, 0.000000}
\pgfsetfillcolor{dialinecolor}
% was here!!!
\pgfsetarrowsend{stealth}
\definecolor{dialinecolor}{rgb}{0.000000, 0.000000, 0.000000}
\pgfsetstrokecolor{dialinecolor}
\draw (14.000000\du,20.000000\du)--(29.000000\du,20.000000\du);
}
% setfont left to latex
\definecolor{dialinecolor}{rgb}{0.000000, 0.000000, 0.000000}
\pgfsetstrokecolor{dialinecolor}
\node at (30.000000\du,20.221250\du){$t$};
\end{tikzpicture}

            \caption{Perfil característico de um recurso consumível.} \label{fig:recurso_consumivel}
        \end{figure}
        
        A Figura \ref{fig:recurso_consumivel} mostra que o recurso $r$ possui inicialmente um valor menor que sua capacidade. Considerando $z_r(0) = z_0$ com $z_0 \in (0; Q_r]$, ao iniciar $a_1$, em $s(a_1)$, $r$ começa a ser consumido por $a_1$, evoluindo como uma função degrau ao longo do tempo. Em $s(a_2)$, inicia-se o consumo de $r$ pela tarefa $a_2$. A quantidade consumida por $a_2$ também varia no tempo como uma função degrau. Quando em $s(a_3)$, inicia-se a tarefa $a_3$ e, simultaneamente, se termina a execução da tarefa $a_2$. Neste instante, inicia-se a produção do recurso $r$, progredindo como uma função degrau ao final de $a_3$. Ao final da execução de cada tarefa, o efeito que cada uma tem sobre a quantidade do recurso consumível $r$ é nulo. Note a diferença entre essa classe de recurso com a classe de recursos reusáveis. Ao lidar com recursos reusáveis, ao término de cada tarefa, é devolvida a quantidade utilizada durante sua execução. Enquanto, ao lidar com recursos consumíveis, não há devoluções para sua quantidade no instante em que a execução da tarefa termina.
        
        São mostrados alguns modelos de funções temporais no Apêndice \ref{app:funcoes}. A seguir serão citados os tipos dos modelos sugeridos no Apêndice \ref{app:funcoes} para a função temporal de produção de um dado recurso consumível $r$: (1) a Figura \ref{fig:funcao_degrau_ascendente} mostra um modelo do tipo degrau; (2) a Figura \ref{fig:funcao_linear_ascendente} mostra um modelo do tipo linear; e, enfim, (3) a Figura \ref{fig:funcao_exponencial_ascendente} mostra um modelo do tipo exponencial. Da mesma forma, serão agora citados os tipos dos modelos sugeridos no Apêndice \ref{app:funcoes} para a função temporal de consumo de um dado recurso consumível $r$: (1) é mostrado na Figura \ref{fig:funcao_degrau_descendente} um modelo do tipo degrau; (2) na Figura \ref{fig:funcao_linear_descendente}, é mostrado um modelo do tipo linear; e, por fim, (3) é mostrado na Figura \ref{fig:funcao_exponencial_descendente} um modelo do tipo exponencial.
        
    
    % 3a seção do capítulo 4
    \section{Tipos de Recurso} \label{sec:tipos_recurso}
    
        Recursos possuem um tipo, podendo ele ser: (1) contínuo, (2) discreto ou (3) unário. 
        
        Primeiramente, em \textit{recursos contínuos}, a capacidade total do recurso é definida por um número pertencente ao conjunto dos números reais estritamente positivos, enquanto sua quantidade corrente é uma representação numérica que pertence ao conjunto dos números reais não-negativos. Assim temos, 
        
        \begin{equation} \label{eq:recurso_continuo}
            z_r: t \in \mathbb{R}_+ \to z \in [0; Q_r] \subset \mathbb{R}_+ \mid Q_r \in \mathbb{R}_+^*
        \end{equation}
        
        Exemplificando \dots \emph{\color{red} (dar exemplo(s) de recursos reusáveis contínuos e recursos consumíveis contínuos. Falar do tipo de funções tbm: \textit{step}, \textit{pulse}, \textit{linear} e \textit{exponential})}
        
        \textit{Recursos discretos} possuem capacidade total definida por um número inteiro estritamente positivo, isto é, um número natural, e quantidade corrente por um número inteiro não-negativo, obtendo:
        
        \begin{equation} \label{eq:recurso_discreto}
            z_r: t \in \mathbb{R}_+ \to z \in [0; Q_r] \subset \mathbb{Z}_+ \mid Q_r \in \mathbb{N}
        \end{equation}
        
        Exemplificando \dots \emph{\color{red} (dar exemplo(s) de recursos reusáveis discretos e recursos consumíveis discretos. Falar do tipo de funções tbm: \textit{step}, \textit{pulse}, \textit{linear} e \textit{exponential})}
        
        E, finalmente, um \textit{recurso unário} sempre possui capacidade total igual à $1$ e sua quantidade corrente pode assumir os valores $0$ ou $1$. Com isso, podemos concluir que a quantidade corrente do recurso informa sua disponibilidade ao longo do tempo: quando $0$, o recurso se encontra indisponível; e, quando $1$, o recurso está disponível. Em outras palavras,
        
        \begin{equation} \label{eq:recurso_unario}
            z_r: t \in \mathbb{R}_+ \to z \in \{0; 1\} 
            \quad\text{e}\quad
            Q_r = 1
        \end{equation}
        
        Exemplificando \dots \emph{\color{red} (dar exemplo(s) de recursos reusáveis unários e recursos consumíveis unários. Falar do tipo de funções tbm: \textit{step}, \textit{pulse}, \textit{linear} e \textit{exponential})}
        
        \dots \emph{\color{red} (dar de uma aplicação com várias tarefas que utilizam recursos de classes e tipos variados.)}
