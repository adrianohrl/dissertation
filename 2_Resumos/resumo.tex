\newpage
% resumo em português
\begin{resumo}
    Este trabalho apresenta o desenvolvimento do pacote baseado em ROS \textit{rqt\_mrta}, o qual fornece um \textit{plugin} de interface gráfica de usuário para a parametrização amigável de arquiteturas para a resolução de problemas de alocação de tarefa em sistema multirrobô. Além disso, em tempo de execução, o \textit{plugin} dispõe elementos gráficos para a supervisão e monitoramento da arquitetura e do sistema multirrobô. 
    
    Devido a pequena quantidade de aproximações genéricas de arquiteturas no ROS, foi desenvolvido o pacote \textit{alliance} que implementa a arquitetura ALLIANCE para testar o \textit{plugin rqt\_mrta}. Em adição, foi elaborada e simulada uma aplicação de patrulha desempenhada por três robôs utilizando a arquitetura do pacote \textit{alliance}. A parametrização da arquitetura do pacote \textit{alliance} e a criação da aplicação foram realizadas através do \textit{plugin rqt\_mrta}. Durante a execução da aplicação, foram analisados os recursos gráficos de supervisão do \textit{rqt\_mrta}. Finalmente, através deste teste, pode-se constatar que o pacote reduziu a complexidade da tarefa de parametrização de arquiteturas de alocação de tarefa e auxilia na supervisão do estado do sistema multirrobô.
    
    \vspace{\onelineskip}
    
    \noindent
    \textbf{Palavras-chaves}: ALLIANCE. Alocação de Tarefa. Arquitetura. ROS. Sistema Multirrobô. 
\end{resumo}
\newpage