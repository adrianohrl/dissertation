%%%%%%%%%%%%%%%%%%%%%%%%%%%%%%%%%%%%%%%%%%%%%%%%%%%%%%%%%%%%%%%%%%%%%%%%%
%
% ALTERAR AS INFORMAÇÕES DESSE BLOCO
%
%%%%%%%%%%%%%%%%%%%%%%%%%%%%%%%%%%%%%%%%%%%%%%%%%%%%%%%%%%%%%%%%%%%%%%%%%
%%%%%%%%%%%%%%%%%%%%%%%%%%%%%%%%%%%%%%%%%%%%%%%%%%%%%%%%%%%%%%%%%%%%%%%%%

% IMPORTANTE: Inserir aqui o tipo do trabalho (mestrado ou doutorado)
%\def \tipo{doutorado}
\def \tipo{mestrado}

% Informações de dados para CAPA e FOLHA DE ROSTO
%\titulo{rqt\_mrta: Uma Interface Gráfica de Usuário baseada em ROS para configuração e supervisão de Arquiteturas de Alocação de Tarefas em Sistema Multirrobô}
%\titulo{rqt\_mrta: Uma interface gráfica para configuração e supervisão de Arquiteturas de Alocação de Tarefa em Sistema Multirrobô no ROS}
%\titulo{rqt\_mrta: Uma GUI para configuração e supervisão de Arquiteturas MRTA no ROS}
%\titulo{rqt\_mrta: Uma aplicação para configuração e supervisão de Arquiteturas MRTA no ROS}
\titulo{rqt\_mrta: Um Pacote ROS para Configuração e Supervisão de Arquiteturas MRTA}
\autor{Adriano Henrique Rossette Leite}
\local{Itajubá}
\data{\today}
\orientador{Prof. Dr. Guilherme Sousa Bastos}
%\coorientador{Prof. Dr. Coorientador}
\instituicao{Universidade Federal de Itajubá}
\def \programaLinhaUm{Programa de Pós-Graduação em}
\def \programaLinhaDois{Engenharia Elétrica}
\def \programa{\programaLinhaUm \space \programaLinhaDois}
\def \areaconcentracao{Automação e Sistemas Elétricos Industriais}

% Data de aprovação
\def \diadeaprovacao{15}
\def \mesdeaprovacao{Dezembro}
\def \anodeaprovacao{2017}

% Banca examinadora - Professores convidados
\def \professorConvidadoUm{Prof. Dr. Edson Prestes}
\def \professorConvidadoDois{Prof. Dr. Laércio Augusto Baldochi Júnior}
%\def \professorConvidadoTres{Prof. Dr. Guilherme Sousa Bastos}

%%%%%%%%%%%%%%%%%%%%%%%%%%%%%%%%%%%%%%%%%%%%%%%%%%%%%%%%%%%%%%%%%%%%%%%%%
%%%%%%%%%%%%%%%%%%%%%%%%%%%%%%%%%%%%%%%%%%%%%%%%%%%%%%%%%%%%%%%%%%%%%%%%%
%%%%%%%%%%%%%%%%%%%%%%%%%%%%%%%%%%%%%%%%%%%%%%%%%%%%%%%%%%%%%%%%%%%%%%%%%
%%%%%%%%%%%%%%%%%%%%%%%%%%%%%%%%%%%%%%%%%%%%%%%%%%%%%%%%%%%%%%%%%%%%%%%%%

\ifthenelse{\equal{\tipo}{mestrado}}{
    \def \tipoTrabalhoUm{Dissertação}
    \def \tipoTrabalhoDois{Mestrado}
    \def \titulacao{Mestre}
}{
    
    \ifthenelse{\equal{\tipo}{doutorado}}{
        \def \tipoTrabalhoUm{Tese}
        \def \tipoTrabalhoDois{Doutorado}
        \def \titulacao{Doutor}
    }{
        \def \tipoTrabalhoUm{Undefined}
        \def \tipoTrabalhoDois{Undefined}
        \def \titulacao{Undefined}
    }
}

\tipotrabalho{\tipoTrabalhoUm \space (\tipoTrabalhoDois)}

% O preambulo deve conter o tipo do trabalho, o objetivo, o nome da instituição e a área de concentração 
\preambulo{\tipoTrabalhoUm \space submetida ao \programa \space como parte dos requisitos para obtenção do Título de \titulacao \space em Ciências em \programaLinhaDois.}

% frase utilizada na folha de aprovação
\def  \aprovacao{\large \tipoTrabalhoUm \space aprovada por banca examinadora em \diadeaprovacao \space de \mesdeaprovacao \space de \anodeaprovacao, conferindo ao autor o título de \textbf {\titulacao \space em Ciências em \programaLinhaDois.}}

% Banca examinadora
\def \bancaexaminadora{\professorConvidadoUm 
    \\ \professorConvidadoDois
    %\\ \professorConvidadoTres
}